\documentclass[draft]{article}

% - style template
\usepackage{base}
\RequirePackage[color=pink]{todonotes}

% - Title
\gdef\theassessment{PHYS4001 - Thesis}
\gdef\thesupervisor{Professor Igor Bray}
\gdef\theinstitution{Curtin University}
\gdef\thestudentid{1834 2884}

\title{Ionisation-with-Excitation Calculations for Electron-Impact Helium
  Collisions within the S-Wave Model}
\author{Thomas Ross}
\date{\today}

% - headers
\pagestyle{fancy}
\fancyhf{}
\rhead{\theauthor}
\chead{}
\lhead{\theassessment}
\rfoot{\thepage}
\cfoot{}
\lfoot{}

% - section labelling
\setcounter{secnumdepth}{3}

\def\frontmatter{%
    \pagenumbering{roman}
    \setcounter{page}{1}
    \renewcommand{\thesection}{\Roman{section}}
}%

\def\mainmatter{%
    \pagenumbering{arabic}
    \setcounter{page}{1}
    \setcounter{section}{0}
    \renewcommand{\thesection}{\arabic{section}}
}%

\def\backmatter{%
    \setcounter{section}{0}
    \renewcommand{\thesection}{\Alph{section}}
}%

% - document
\begin{document}

% cover page

\begin{titlepage}
  \begin{flushleft}
    \theinstitution \hfill \theassessment
  \end{flushleft}
  \hrule
  \begin{center}
    {\huge{\thetitle}\par}
    {\rule[1.0pt]{8.5cm}{0.4pt}\par}
    {\Large{\theauthor}\par}
    {\Large{Supervised by \thesupervisor}\par}
  \end{center}
  \hrule
  \vspace*{\fill}
  \begin{center}
    \todo[inline]{Write abstract.}
  \end{center}
\end{titlepage}

\clearpage

\frontmatter

\section*{Declaration}

\todo[inline]{Write declaration.}

\section*{Acknowledgements}

\todo[inline]{Write acknowledgements.}

\clearpage

\tableofcontents

\clearpage

\listoffigures

\clearpage

\listoftables

\clearpage

\section*{List of Abbreviations}

\begin{itemize}
\item[TCS:] total cross section
\item[SDCS:] single-differential cross section
\item[DDCS:] double-differential cross section
\item[TDCS:] triple-differential cross section
\item[TICS:] total ionisation cross section
\item[CCC:] convergent close-coupling
\item[CCC($N$):] convergent close-coupling calculation performed with $N$
  one-electron basis states
\item[CCC($C, N$):] convergent close-coupling calculation performed with $C$ core
  states and $N$ one-electron basis states
\item[CCC($C, N, \lambda$):] convergent close-coupling calculation performed with
  $C$ core states, and $N$ one-electron basis states with exponential fall-off
  parameter $\lambda$
\item[ECS:] exterior complex scaling
\item[PECS:] propagating exterior complex scaling
\end{itemize}

\clearpage

\mainmatter

\section{Introduction}
\label{sec:in}

\todo[inline]{Describe utility of Electron-Impact Helium scattering processes.}

\subsection{Electron-Impact Helium Scattering Processes}
\label{sec:in-proc}

\todo[inline]{Describe elastic, excitation and ionisation scattering processes.}

\todo[inline]{Describe auto-ionisation process for excited Helium.}

\todo[inline]{Describe atomic term symbols (in context of Helium), and discuss
  Helium states.}

\subsection{Experimental Review}
\label{sec:in-exp}

\subsection{Theoretical Review}
\label{sec:in-th}

\todo[inline]{Discuss early development of CCC method for Electron-impact
  Hydrogen scattering (elastic, excitation, ionisation).}

\todo[inline]{Discuss extension of CCC method to three-electon systems.}

\todo[inline]{Discuss challenges encountered and overcome in obtaining accurate
  DCS's for ionisation processes.}

\todo[inline]{Discuss decision to use S-wave model.}

\todo[inline]{Discuss early CCC data for Helium TICS.}

\todo[inline]{Discuss PECS data demonstrating agreement with CCC data for
  TICS-without-excitation but not for TICS-with-excitation.}

\section{Theory}
\label{sec:th}

\subsection{Convergent Close-Coupling Method for an Atomic Target}
\label{sec:th-ccc}

\subsubsection{Laguerre Basis}
\label{sec:th-ccc-lag}

\subsubsection{Target States}
\label{sec:th-ccc-target}

\subsubsection{Total Wavefunction}
\label{sec:th-ccc-total}

\subsubsection{Convergent Close-Coupling Equations}
\label{sec:th-ccc-eq}

\subsection{Scattering Statistics}
\label{sec:th-ccc-stat}

\subsubsection{Scattering Amplitudes}
\label{sec:th-ccc-amp}

\subsubsection{Ionisation Cross-Sections}
\label{sec:th-ccc-ion}

\subsection{Considerations for a Helium Target}
\label{sec:th-he}

\subsubsection{Partially Frozen-Core Model}
\label{sec:th-he-frozen}

\subsubsection{Auto-Ionising Target States}
\label{sec:th-he-auto}

\section{Results}
\label{sec:re}

\subsection{Helium Target States}
\label{sec:re-he}

\todo[inline]{Discuss major-configuration purity of states as function of
  exponential fall-off.}

\todo[inline]{Figure of major-configuration purity for doubly-excited states.}

\todo[inline]{Discuss interference of doubly-excited and continuum states
  (auto-ionisation).}

\todo[inline]{Figure of Helium energy spectrum(s) and auto-ionisation threshold.}

\todo[inline]{Discuss improvements in fidelity of target states and increase in
  computational cost with increasing number of core states.}

\subsection{Total Ionisation-without-Excitation Cross-Sections}
\label{sec:re-ticswout}

\todo[inline]{Discuss agreement of CCC and PECS data for TICS-without-excitation.}

\todo[inline]{Figure of CCC and PECS data for TICS-without-excitation.}

\subsection{Total Ionisation-with-Excitation Cross-Sections}
\label{sec:re-ticsw}

\todo[inline]{Discuss difficulty associated with the small magnitude of
  TICS-with-excitation.}

\todo[inline]{Figure of elastic, TICS-with-excitation and
  TICS-without-excitation, demonstrating magnitude difference.}

\todo[inline]{Discuss how convergence is attained in multi-parameter setting
  (increasing the number of core states for a fixed number of one-electron basis
  states).}

\todo[inline]{Figure of TICS-with-excitation for increasing number of core
  states, demonstrating convergence.}

\todo[inline]{Discuss sensitivity of TICS-with-excitation to exponential
  fall-off parameter / target state fidelity.}

\todo[inline]{Figure of TICS-with-excitation for varying exponential fall-off
  demonstrating variation.}

\todo[inline]{Discuss difficulty in removing pseudoresonances from
  TICS-with-excitation.}

\todo[inline]{Discuss decreasing magnitude TICS-with-excitation up to a certain
  number of one-electron-basis states, and increasing magnitude past this
  point. Mention how it may be similar to variations with exponential fall-off
  parameter, being affected by fidelity of target states.}

\todo[inline]{Figure of TICS-with-excitation for increasing number of
  one-electron basis states, demonstrating suggestion of convergence in
  magnitude then also failure to converge.}

\section{Conclusion}
\label{sec:co}

\clearpage

\bibliography{references}

\clearpage

\listoftodos

\end{document}
