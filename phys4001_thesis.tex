\documentclass[]{article}

% - style template
\usepackage{base}
\RequirePackage[color=pink]{todonotes}

% - Title
\gdef\theassessment{PHYS4001 - Thesis}
\gdef\thesupervisor{Professor Igor Bray}
\gdef\theinstitution{Curtin University}
\gdef\thestudentid{1834 2884}

\title{Ionisation-with-Excitation Calculations for Electron-Impact Helium
  Collisions within the S-Wave Model}
\author{Thomas Ross}
\date{\today}

% - headers
\pagestyle{fancy}
\fancyhf{}
\rhead{\theauthor}
\chead{}
\lhead{\theassessment}
\rfoot{\thepage}
\cfoot{}
\lfoot{}

% - section labelling
\setcounter{secnumdepth}{3}

\def\frontmatter{%
    \pagenumbering{roman}
    \setcounter{page}{1}
    \renewcommand{\thesection}{\Roman{section}}
}%

\def\mainmatter{%
    \pagenumbering{arabic}
    \setcounter{page}{1}
    \setcounter{section}{0}
    \renewcommand{\thesection}{\arabic{section}}
}%

\def\backmatter{%
    \setcounter{section}{0}
    \renewcommand{\thesection}{\Alph{section}}
}%

% - document
\begin{document}

% cover page

\begin{titlepage}
  \begin{flushleft}
    \theinstitution \hfill \theassessment
  \end{flushleft}
  \hrule
  \begin{center}
    {\huge{\thetitle}\par}
    {\rule[1.0pt]{8.5cm}{0.4pt}\par}
    {\Large{\theauthor}\par}
    {\Large{Supervised by \thesupervisor}\par}
  \end{center}
  \hrule
  \vspace*{\fill}
  \begin{center}
    \todo[inline]{Write abstract.}
  \end{center}
\end{titlepage}

\clearpage

\frontmatter

\section*{Declaration}

\todo[inline]{Write declaration.}

\section*{Acknowledgements}

\todo[inline]{Write acknowledgements.}

\clearpage

\tableofcontents

\clearpage

\listoffigures

\clearpage

\listoftables

\clearpage

\section*{List of Abbreviations}

\begin{itemize}
\item[TCS:]
  total cross section
\item[SDCS:]
  single-differential cross section
\item[DDCS:]
  double-differential cross section
\item[TDCS:]
  triple-differential cross section
\item[TICS:]
  total ionisation cross section
\item[TIECS:]
  total ionisation-with-excitation cross section; alternatively written as
  TICS-with-excitation
\item[CCC:]
  convergent close-coupling
\item[CCC($N$):]
  convergent close-coupling calculation performed with $N$ one-electron basis
  states
\item[CCC($C, N$):]
  convergent close-coupling calculation performed with $C$ core states and $N$
  one-electron basis states
\item[CCC($C, N, \lambda$):]
  convergent close-coupling calculation performed with $C$ core states, and $N$
  one-electron basis states with exponential fall-off parameter $\lambda$
\item[ECS:]
  exterior complex scaling
\item[PECS:]
  propagating exterior complex scaling
\end{itemize}

\clearpage

\section*{List of Notation}

\begin{itemize}
\item[$\ket*{\varphi_{i}}$]
  Laguerre basis states

\item[$\hat{A}$]
  anti-symmetriser operator
\item[$\hat{P}_{i, j}$]
  pairwise exchange operators

\item[$\hat{H}_{T}$]
  target Hamiltonian operator
\item[$\hat{K}_{m}$]
  target electron kinetic operators
\item[$\hat{V}_{m}$]
  target electron potential operators
\item[$\hat{V}_{m, n}$]
  target electron-electron potential operators
\item[$\hat{H}_{T, e}$]
  target Hamiltonian operator, restricted to one target electron

\item[$\ket*{\phi_{i}}$]
  one-electron atomic orbitals
\item[$\ket*{\chi_{i}}$]
  one-electron spin orbitals
\item[$\ket*{\chi_{\lrsq{a_{1}, \dotsc, a_{n}}}}$]
  Slater determinants

\item[$\ket*{\Phi_{n}}$ / $\epsilon_{n}$]
  target (states / energies)
\item[$\ket*{\Phi_{n}^{\lr{N}}}$ / $\epsilon_{n}^{\lr{N}}$]
  target (pseudostates / pseudoenergies), calculated with $N$ one-electron basis
  states
\item[$\ket*{\Phi_{n}^{\lr{C, N}}}$ / $\epsilon_{n}^{\lr{C, N}}$]
  target (pseudostates / pseudoenergies), calculated with $C$ core states, and
  $N$ one-electron basis states

\item[$\hat{I}_{T}$]
  target states projection operator
\item[$\hat{I}_{T}^{\lr{N}}$]
  target pseudostates projection operator, calculated with $N$ one-electron
  basis states
% \item[$\hat{I}_{T}^{\lr{C, N}}$]
%   target pseudostates projection operator, calculated with $C$ core states, and
%   $N$ one-electron basis states

\item[$\ket*{\vb{k}_{\alpha}}$ / $\tfrac{1}{2}k_{\alpha}^{2}$]
  projectile (states / energies), taking the form of continuum waves

\item[$\hat{H}$]
  total Hamiltonian operator
\item[$\hat{K}_{0}$]
  projectile electron kinetic operator
\item[$\hat{V}_{0}$]
  projectile electron-nuclei potential operator
\item[$\hat{V}_{0, m}$]
  projectile electron-target electron potential operators

\item[$\ket*{\Psi}$]
  total wavefunction
\item[$E$]
  total energy
\item[$\ket*{\psi}$]
  unsymmetrised total wavefunction
\item[$\ket*{F_{n}^{\lr{N}}}$]
  multichannel weight functions, calculated with $N$ one-electron basis states
\item[$\ket*{\Psi^{\lr{N}}}$ / $\ket*{\psi^{\lr{N}}}$]
  multichannel-expanded (total wavefunction / unsymmetrised total wavefunction)

% % -- lippmann-schwinger derivation
% \item[$\hat{H}_{A}$]
%   asymptotic Hamiltonian operator, which is unbounded
% \item[$\hat{V}$]
%   potential operator, which is bounded
% \item[$\ket*{\Omega_{\alpha}}$ / $\varepsilon_{\alpha}$]
%   asymptotic (eigenstates / eigenenergies)
% \item[$\ket*{\Omega_{\alpha}^{\lr{E}}}$]
%   on-shell asymptotic eigenstates
% \item[$\hat{G}_{\lr{E}}$]
%   Green's operator
% \item[$\hat{T}$]
%   the $\hat{T}$ operator
% \item[$\hat{K}$]
%   the $\hat{K}$ operator

\item[$\hat{H}_{A}$]
  asymptotic Hamiltonian operator, which is unbounded
\item[$\hat{V}$]
  anti-symmetrised potential operator, which is bounded
\item[$\ket*{\Phi_{\alpha} \vb{k}_{\alpha}}$ / $\varepsilon_{\alpha}$]
  asymptotic (states / energies)
\item[$\ket*{\Phi_{n_{\alpha}}^{\lr{N}} \vb{k}_{\alpha}}$
  / $\varepsilon_{\alpha}^{\lr{N}}$]
  asymptotic (pseudostates / pseudoenergies), calculated with $N$ one-electron
  basis states
\item[$\hat{T}$]
  the $\hat{T}$ operator
\item[$\hat{K}$]
  the $\hat{K}$ operator
\item[$k_{n}$]
  on-shell projectile momenta

\item[$f_{\alpha, \beta}$]
  scattering amplitudes, equivalently:
  \begin{itemize}
  \item[$=$]
    $f_{\alpha, \beta}\lr{\vb{k}_{\alpha}, \vb{k}_{\beta}}$
  \item[$=$]
    $\mel*{\vb{k}_{\alpha}\Phi_{\alpha}}{\hat{V}}{\Psi_{\beta}}$
  \item[$=$]
    $\mel*{\vb{k}_{\alpha}\Phi_{\alpha}}{\hat{T}}{\Phi_{\beta}\vb{k}_{\beta}}$
  \end{itemize}
\item[$f_{i}^{\lr{N}}$]
  elastic scattering amplitudes, calculated with $N$ one-electron basis states
\item[$f_{f, i}^{\lr{N}}$]
  discrete excitation scattering amplitudes, calculated with $N$ one-electron
  basis states
\item[$f_{\alpha, i}^{\lr{N}}$]
  (unsymmetrised) ionisation scattering amplitudes, calculated with $N$
  one-electron basis states
\item[$F_{\alpha, i}^{\lr{N}}$]
  anti-symmetrised ionisation scattering amplitudes, calculated with $N$
  one-electron basis states
\item[$\sigma_{\alpha, \beta}$]
  partial cross sections, equivalently:
  \begin{itemize}
  \item[$=$]
    $\sigma_{\alpha, \beta}\lr{\vb{k}_{\alpha}, \vb{k}_{\beta}}$
  \item[$=$]
    $\dfrac{k_{\alpha}}{k_{\beta}}\lrabs{f_{\alpha, \beta}}^{2}$
  \item[$=$]
    $
    \dfrac{k_{\alpha}}{k_{\beta}}
    \lrabs
    {
      \mel*
      {\vb{k}_{\alpha} \Phi_{\alpha}}
      {\hat{T}}
      {\Phi_{\beta} \vb{k}_{\beta}}
    }^{2}
    $
  \end{itemize}

\item[$\sigma_{\mathrm{T}; i}^{\lr{N}}$]
  total cross section for a given initial asymptotic state, calculated with $N$
  one-electron basis states
\item[$\sigma_{\mathrm{I}; i}^{\lr{N}}$]
  total ionisation cross section for a given initial asymptotic state,
  calculated with $N$ one-electron basis states
\item[$\sigma_{\mathrm{I}; n_{f}, i}^{\lr{N}}$]
  ionisation cross section for given final and initial asymptotic states,
  calculated with $N$ one-electron basis states
\end{itemize}

\clearpage

\mainmatter

\section{Introduction}
\label{sec:in}

\todo[inline]{Describe utility of Electron-Impact Helium scattering processes.}

\subsection{Helium Atom}
\label{sec:in-he}

\todo[inline]{Describe atomic term symbols (in context of Helium), and discuss
  Helium states.}

The helium atom consists of two electrons bound electromagnetically to a nucleus
containing two protons and, restricting our attention to stable isotopes, either
one (helium-3) or two (helium-4) neutrons.
As the helium atom is light, it's (non-excited) quantum states are well defined
by the following quantum numbers:
\begin{itemize}
\item $S$: total spin,
\item $L$: orbital angular momentum,
\item $J$: total angular momentum,
\end{itemize}
and can be compactly written with the term symbol ${}^{2S + 1}L_{J}$, where
$2S + 1$ is the spin multiplicity, in accordance with the $LS$-coupling scheme.
A specific electron configuration of the helium atom can be prepended to the
term symbol when desired, being written in the form
$n_{1}\ell_{1} n_{2}\ell_{2} {}^{2S + 1}L_{J}$, where $n_{1}, n_{2}$ are the
principal quantum numbers, and $\ell_{1}, \ell_{2}$ are the orbital angular
momentum quantum numbers for each electron.
In the context of the S-wave model, we consider only the quantum states of the
helium atom with $L = 0$: the singlet state ${}^{1}\mathrm{S}_{0}$, and the
triplet state ${}^{3}\mathrm{S}_{1}$.

\subsection{Electron-Impact Helium Scattering Processes}
\label{sec:in-proc}

\todo[inline]{Describe elastic, excitation and ionisation scattering processes.}

The impact of an electron projectile on a helium target can lead to a number of
different scattering processes:
\begin{alignat*}{4}
  &
  \mathrm{e}^{-}
  +
  \mathrm{He}
  &
  {}\to{}
  &
  \mathrm{e}^{-}
  +
  \mathrm{He}
  &
  \quad\quad
  &
  \text{(elastic)}
  \\
  &
  &
  {}\to{}
  &
  \mathrm{e}^{-}
  +
  \mathrm{He}^{'}
  &
  &
  \text{(single-excitation)}
  \\
  &
  &
  {}\to{}
  &
  \mathrm{e}^{-}
  +
  \mathrm{He}^{''}
  &
  &
  \text{(double-excitation)}
  \\
  &
  &
  {}\to{}
  &
  2
  \mathrm{e}^{-}
  +
  \mathrm{He}^{+}
  &
  &
  \text{(ionisation-without-excitation)}
  \\
  &
  &
  {}\to{}
  &
  2
  \mathrm{e}^{-}
  +
  \mathrm{He}^{+'}
  &
  &
  \text{(ionisation-with-excitation)}
  \\
  &
  &
  {}\to{}
  &
  3
  \mathrm{e}^{-}
  +
  \mathrm{He}^{++}
  &
  &
  \text{(double-ionisation / breakup)}
  .
\end{alignat*}

\todo[inline]{Describe auto-ionisation process for excited Helium.}

Helium also exhibits the phenomena of auto-ionisation, in which a doubly-excited
helium state may spontaneously eject one electron:
\begin{alignat*}{4}
  &
  \mathrm{He}^{''}
  &
  {}\to{}
  &
  \mathrm{e}^{-}
  +
  \mathrm{He}^{+}
  &
  \quad\quad
  &
  \text{(auto-ionisation-without-excitation)}
  \\
  &
  &
  {}\to{}
  &
  \mathrm{e}^{-}
  +
  \mathrm{He}^{+'}
  &
  \quad\quad
  &
  \text{(auto-ionisation-with-excitation)}
  .
\end{alignat*}
The resulting helium ion may be in the ground state or an excited state,
subject to the constraint of energy conservation.
The interference between the discrete, doubly-excited states of helium and the
unbounded, continuum states of the auto-ionised system is non-trivial and has
significant consequences for the double-excitation scattering process.

\todo[inline]{Reference Fano regarding auto-ionisation.}

\subsection{Experimental Review}
\label{sec:in-exp}

\subsection{Theoretical Review}
\label{sec:in-th}

\todo[inline]{Discuss early development of CCC method for Electron-impact
  Hydrogen scattering (elastic, excitation, ionisation).}

\todo[inline]{Discuss extension of CCC method to three-electon systems.}

\todo[inline]{Discuss challenges encountered and overcome in obtaining accurate
  DCS's for ionisation processes.}

\todo[inline]{Discuss decision to use S-wave model.}

\todo[inline]{Discuss early CCC data for Helium TICS.}

\todo[inline]{Discuss PECS data demonstrating agreement with CCC data for
  TICS-without-excitation but not for TICS-with-excitation.}

\clearpage

\section{Theory}
\label{sec:th}

We shall present a brief derivation of the Convergent Close-Coupling (CCC)
method for generalised electron-projectile atomic/ionic-target scattering,
similar in form to the derivations presented in \cite{BRAY19951, AJP_BRAY1996}.
The specific considerations for the application of the CCC method to the case of
electron-impact helium (e-He) scattering is discussed in \autoref{sec:th-he}.
We shall focus on the treatment of target ionisation, both with and without
excitation, by consideration of the ionisation amplitudes within the CCC method.

\subsection{Convergent Close-Coupling Method for an Atomic Target}
\label{sec:th-ccc}

In brief, the CCC method utilises the method of basis expansion to numerically
solve the Lippmann-Schwinger equation in a momentum-space representation, for a
projectile-target system, to yield the transition amplitudes, which are checked
for convergence as the size of the basis is increased.
The scattering statistics can then be extracted from the transistion amplitudes.

The rate of convergence, depends on many factors, such as the complexity of the
target structure, the coupling between transition channels, and the choice of
basis used in the expansion.
With the selection of an appropriate basis, the unbounded continuum waves can be
represented (to a sufficient accuracy) by a finite number of basis states, which
allows ionisation amplitudes to be treated in a similar manner to discrete
excitation amplitudes within the CCC method.
A Laguerre basis is well-suited to this task; the benefits of this basis are
discussed in further detail in \cite[5-9]{BRAY19951}.

\subsubsection{Laguerre Basis}
\label{sec:th-ccc-lag}

To describe the target structure, the CCC method utilises a Laguerre basis
$\lrset{\ket{\varphi_{i}}}_{i = 1}^{\infty}$ for the Hilbert space
$L^{2}\lr{\real^{3}}$, for which the coordinate-space representation is of the
form
\begin{equation}
  \label{eq:laguerre-basis}
  \bra{\vb{r}}
  \ket{\varphi_{i}}
  =
  \varphi_{i}\lr{r, \Omega}
  =
  \tfrac{1}{r}
  \xi_{k_{i}, l{i}}\lr{r}
  Y_{l_{i}}^{m_{i}}\lr{\Omega}
  ,
\end{equation}
where $Y_{l_{i}}^{m_{i}}\lr{\Omega}$ are the spherical harmonics, and where
$\xi_{k_{i}, l_{i}}\lr{r}$ are the Laguerre radial basis functions, which are of
the form
\begin{equation}
  \label{eq:laguerre-radial-basis}
  \xi_{k, l}\lr{r}
  =
  \sqrt
  {
    \dfrac
    {
      \lambda_{l}
      \lr{k - 1}!
    }
    {
      \lr{2l + 1 + k}!
    }
  }
  \lr{\lambda_{l} r}^{l + 1}
  \exponential\lr[\big]{- \tfrac{1}{2} \lambda_{l} r}
  L_{k - 1}^{2l + 2}\lr{\lambda_{l} r}
  ,
\end{equation}
where $\lambda_{l}$ is the exponential fall-off, for each $l$, and where
$L_{k - 1}^{2l + 2}\lr{\lambda_{l} r}$ are the associated Laguerre polynomials.
Note that we must have that
$k_{i} \in \lrset{1, 2, \dotsc}$,
$l_{i} \in \lrset{0, 1, \dotsc}$ and
$m_{i} \in \lrset{-\ell_{i}, \dotsc, \ell_{i}}$, for each
$i \in \lrset{1, 2, \dotsc}$.

This Laguerre basis is utilised due to: the Laguerre basis functions
$\lrset{\varphi_{i}\lr{r, \Omega}}_{i = 1}^{\infty}$ forming a complete basis
for the Hilbert space $L^{2}\lr{\real^{3}}$, the short-range and long-range
behaviour of the radial basis functions being well suited to describing bound
target states and providing a basis for expanding continuum states in, and
because it allows the matrix representation of numerous operators to be
calculated analytically.

Practically, we cannot utilise a basis of infinite size.
Hence, we truncate the Laguerre radial basis
$\lrset{\xi_{k, l}\lr{r}}_{k = 1}^{N_{l}}$ to a certain number of radial basis
functions $N_{l}$, for each $l$, and we also truncate
$l \in \lrset{0, \dotsc, l_{\max}}$,
limiting the maximum angular momentum we consider in our basis.
Hence, for a given value of $m$, we have a basis size of
\begin{equation}
  \label{eq:basis-size}
  N
  =
  \sum_{l = 0}^{l_{\max}}
  N_{l}
  .
\end{equation}
In the limit as $N \to \infty$, the truncated basis will tend towards
completeness, and it is in this limit that we discuss the convergence of the
Convergent Close-Coupling method.
We have presented the Laguerre basis with full generality, however we note that
in the S-wave model we have $l_{\max} = 0$, which allows for the simplification
of numerous expressions and computations.

\subsubsection{Target States}
\label{sec:th-ccc-target}

Possessing now a suitable basis to work with, we proceed to represent the
target in this basis by the method of basis expansion.
Firstly, we note that electrons are indistinguishable fermionic particles; that
is, no two electrons can be distinguished from each other, and they must satisfy
Pauli's exclusion principle - that an electron state cannot be occupied by more
than one electron.
Since electrons are indistinguishable, we might naively suppose that the space
of states consisting of $n$ electrons is simply the $n$-th tensor power of the
one-electron space, $T^{n}\lr{\mathcal{H}}$, defined by
\begin{equation}
  \label{eq:tensor-power}
  T^{n}\lr{\mathcal{H}}
  =
  \lrset
  {
    \ket{\psi_{1}}
    \otimes
    \dots
    \otimes
    \ket{\psi_{n}}
    :
    \ket{\psi_{1}},
    \dotsc,
    \ket{\psi_{n}}
    \in \mathcal{H}
  }
  ,
\end{equation}
where $\mathcal{H}$ is the space of one-electron states.
However this fails to account for Pauli's exclusion principle, since any
one-electron state may be occupied up to $n$ times.
Hence, the space of states consisting of $n$ electrons is instead defined to be
the quotient space $\Lambda^{n}\lr{\mathcal{H}}$ of $T^{n}\lr{\mathcal{H}}$ by
$\mathcal{D}^{n}$,
\begin{equation}
  \label{eq:exterior-power}
  \Lambda^{n}\lr{\mathcal{H}}
  =
  T^{n}\lr{\mathcal{H}}
  /
  \mathcal{D}^{n}
  ,
\end{equation}
where $\mathcal{D}^{n} \subset T^{n}\lr{\mathcal{H}}$ is the subspace of tensor
products which contain any one-electron state more than once.
The space $\Lambda^{n}\lr{\mathcal{H}}$ is known as the $n$-th exterior power of
$\mathcal{H}$, and is identifiable as the subspace of $T^{n}\lr{\mathcal{H}}$
consisting of anti-symmetric tensors.
Note that we shall adopt the following notation for tensor products
\begin{equation}
  \label{eq:tensor-product-notation}
  \ket{\psi_{1}, \dotsc, \psi_{n}}
  =
  \ket{\psi_{1}}
  \otimes
  \dots
  \otimes
  \ket{\psi_{n}}
\end{equation}
and the following notation for anti-symmetric tensor products
\begin{equation}
  \label{eq:anti-symmetric-tensor-product-notation}
  \ket{\lrsq{\psi_{1}, \dotsc, \psi_{n}}}
  =
  \ket{\psi_{\lrsq{1, \dotsc, n}}}
  =
  \sqrt{n!}
  \hat{A}
  \ket{\psi_{1}, \dotsc, \psi_{n}}
\end{equation}
where $\hat{A} : T^{n}\lr{\mathcal{H}} \to \Lambda^{n}\lr{\mathcal{H}}$ is the
anti-symmetriser operator which we define to be of the form
\begin{equation}
  \label{eq:anti-symmetriser}
  \hat{A}
  \ket{\psi_{1}, \dotsc, \psi_{n}}
  =
  \dfrac{1}{n!}
  \sum_{\sigma \in S_{n}}
  \mathrm{sgn}\lr{\sigma}
  \ket{\psi_{\sigma\lr{1}}, \dotsc, \psi_{\sigma\lr{n}}}
  ,
\end{equation}
where $S_{n}$ is the symmetric group on $n$ elements, the sum is taken over all
permutations $\sigma \in S_{n}$, and where $\mathrm{sgn}\lr{\sigma}$ is the
signature of the permutation $\sigma$.
It follows from this construction that
\begin{equation}
  \label{eq:anti-symmetry}
  \ket{\psi_{\lrsq{a_{1}, \dotsc, a_{n}}}}
  =
  0
  \qq{if any}
  a_{i} = a_{j}
  ,
\end{equation}
hence satisfying Pauli's exclusion principle.
Furthermore, we have that
\begin{equation}
  \label{eq:pairwise-exchange}
  \hat{P}_{i, j}
  \ket{\psi_{\lrsq{1, \dotsc, n}}}
  =
  -
  \ket{\psi_{\lrsq{1, \dotsc, n}}}
  ,
\end{equation}
where $\hat{P}_{i, j}$ is the pairwise exchange operator, permuting the states
$\ket*{\psi_{i}}$ and $\ket*{\psi_{j}}$.
We note that in this context, the states $\ket*{\psi_{i}}$ include both coordinate
and spin states.

It follows that for an atomic/ionic target, consisting of $n_{\mathrm{e}}$
electrons, the space of target states is of the form
$\mathcal{H}_{T} = \Lambda^{n_{\mathrm{e}}}\lr{\mathcal{H}}$.
We shall adopt the convention that operators which act on the $m$-th electron
space (including the projectile electron), will be indexed by $m$, for
$m = 0, 1, \dotsc, n_{\mathrm{e}}$, with $m = 0$ indexing the projectile
electron space.

\paragraph{Target Hamiltonian}
\label{sec:th-ccc-target-hamil}

The target Hamiltonian, for an atomic/ionic target with $n_{\mathrm{e}}$
electrons, is of the form
\begin{equation}
  \label{eq:target-hamiltonian}
  \hat{H}_{T}
  =
  \sum_{m = 1}^{n_{\mathrm{e}}}
  \hat{K}_{m}
  +
  \sum_{m = 1}^{n_{\mathrm{e}}}
  \hat{V}_{m}
  +
  \sum_{m = 1}^{n_{\mathrm{e}}}
  \sum_{n = m + 1}^{n_{\mathrm{e}}}
  \hat{V}_{m, n}
  ,
\end{equation}
where $\hat{K}_{m}$ and $\hat{V}_{m}$ are the target electron kinetic and
electron-nuclei potential operators, for $m = 1, \dotsc, n_{\mathrm{e}}$, and
where $\hat{V}_{m, n}$ are the electron-electron potential operators, for
$m, n = 1, \dotsc, n_{\mathrm{e}}$.

\paragraph{Target Diagonalisation}
\label{sec:th-ccc-target-diag}

The target Hamiltonian, restricted to just one target electron,
\begin{equation}
  \label{eq:target-1e-hamiltonian}
  \hat{H}_{T, \mathrm{e}}^{}
  =
  \hat{K}_{1}
  +
  \hat{V}_{1}
  ,
\end{equation}
is expanded in a Laguerre basis $\lrset{\ket{\varphi_{i}}}_{i = 1}^{N}$ and
diagonalised to yield a set of one-electron atomic orbitals
$\lrset{\ket*{\phi_{i}^{\lr{N}}}}_{i = 1}^{N}$ which are orthonormal and
satisfy
\begin{equation}
  \label{eq:target-1e-orbitals}
  \bra*{\phi_{i}^{\lr{N}}}
  \hat{H}_{T, \mathrm{e}}
  \ket*{\phi_{j}^{\lr{N}}}
  =
  \varepsilon_{i}^{\lr{N}}
  \delta_{i, j}
  .
\end{equation}
From these one-electron atomic orbitals, we generate a set of one-electron spin
orbitals $\lrset{\ket*{\chi_{i}^{\lr{N}}}}_{i = 1}^{2N}$ for which
$\ket*{\chi_{2i - 1}^{\lr{N}}}$ and $\ket*{\chi_{2i}^{\lr{N}}}$ both correspond
to $\ket*{\phi_{i}^{\lr{N}}}$ but have spin projection $\tfrac{1}{2}$ and
$-\tfrac{1}{2}$ respectively.
These one-electron spin orbitals are then combined to construct Slater
determinants; for any selection of $n_{\mathrm{e}}$ one-electron spin orbitals
$\ket*{\chi_{a_{1}}^{\lr{N}}}, \dotsc, \ket*{\chi_{a_{n_{\mathrm{e}}}}^{\lr{N}}}
\in \lrset{\ket*{\chi_{i}^{\lr{N}}}}_{i = 1}^{2N}$, the Slater determinant of
these spin orbitals is of the form
\begin{equation}
  \label{eq:target-slater-determinant}
  \ket*{\chi_{\lrsq{a_{1}, \dotsc, a_{n_{\mathrm{e}}}}}^{\lr{N}}}
  =
  \sqrt{n_{\mathrm{e}}!}
  \hat{A}
  \ket*
  {
    \chi_{a_{1}}^{\lr{N}},
    \dotsc,
    \chi_{a_{n_{\mathrm{e}}}}^{\lr{N}}
  }
  =
  \dfrac{1}{\sqrt{n_{\mathrm{e}}!}}
  \sum_{\sigma \in S_{n_{\mathrm{e}}}}
  \mathrm{sgn}\lr{\sigma}
  \ket*
  {
    \chi_{a_{\sigma\lr{1}}}^{\lr{N}}
    ,
    \dots
    ,
    \chi_{a_{\sigma\lr{n_{\mathrm{e}}}}}^{\lr{N}}
  }
  ,
\end{equation}
as per \eqref{eq:anti-symmetric-tensor-product-notation} and
\eqref{eq:anti-symmetriser}.
We note that Slater determinants are anti-symmetric under pairwise exchange of
any two orbitals, and are zero if constructed with two spin orbitals in the same
state.
Hence they adhere to Pauli's exclusion principle and are indeed elements
of $\mathcal{H}_{T} = \Lambda^{n_{\mathrm{e}}}\lr{\mathcal{H}}$.

The true target states
$\lrset{\ket{\Phi_{\alpha}}} \in \mathcal{H}_{T}$ are then
approximated by expanding the full target Hamiltonian $\hat{H}_{T}$ in a basis
of Slater determinants,
\begin{equation}
  \label{eq:target-slater-determinant-basis}
  \lrset[\big]
  {
    \ket*{\chi_{\lrsq{a_{1}, \dotsc, a_{n_{\mathrm{e}}}}}^{\lr{N}}}
    :
    a_{1}, \dotsc, a_{n_{\mathrm{e}}}
    \in
    \lrset{1, \dotsc, 2N}
  }
  ,
\end{equation}
and diagonalising to yield a set of target pseudostates
$\lrset{\ket*{\Phi_{n}^{\lr{N}}}}_{n = 1}^{N_{T}}$ which are orthonormal and
satisfy
\begin{equation}
  \label{eq:target-states}
  \bra*{\Phi_{i}^{\lr{N}}}
  \hat{H}_{T}
  \ket*{\Phi_{j}^{\lr{N}}}
  =
  \epsilon_{i}^{\lr{N}}
  \delta_{i, j}
  ,
\end{equation}
where $\epsilon_{n}^{\lr{N}}$ is the pseudoenergy corresponding to the
pseudostate $\ket*{\Phi_{n}^{\lr{N}}}$.
Note that the number of target pseudostates $N_{T}$ depends on the number of
Slater determinants utilised in the expansion of $\hat{H}_{T}$.
Note also that the $\lr{N}$ superscript has been introduced to indicate that
these are not true eigenstates of the target Hamiltonian, only of its
representation in the truncated Laguerrre basis, and that these pseudostates and
their pseudoenergies are dependent on the size of the Laguerre basis utilised.

We remark that the target pseudostates will be expressed as a linear combination
of Slater determinants of the form
\begin{equation}
  \label{eq:target-states-expansion}
  \ket*{\Phi_{n}^{\lr{N}}}
  =
  \sum_{1 \leq a_{1} < \dotsc < a_{n_{\mathrm{e}}} \leq 2N}
  D_{n}^{a_{1}, \dotsc, a_{n_{\mathrm{e}}}}
  \ket*{\chi_{\lrsq{a_{1}, \dotsc, a_{n_{\mathrm{e}}}}}^{\lr{N}}}
\end{equation}
where $D_{n}^{a_{1}, \dotsc, a_{n_{\mathrm{e}}}}$ are the expansion
coefficients.
The summation indices are taken over only linearly independent Slater
determinants; a consequence of the anti-symmetry of the Slater determinants
under pairwise exchange of any two orbitals.
In this context, the Slater determinants are often referred to as
(electron) configurations, being the assembly of $n_{\mathrm{e}}$ one-electron
configurations.

Using all possible Slater determinants in the expansion is often computationally
infeasible, as the number of determinants scales as
$\binom{2N}{n_{\mathrm{e}}}$.
A common method of mitigating this computational hindrance, which we shall
utilise, is to use only a subset of the full set of Slater determinants, for
which the span of this subset remains sufficiently able to describe the target
pseudostates to a required degree of accuracy.
Specifically, the target orbitals are partitioned into a core set and valence
set of orbitals, with the core orbitals being limited to a much smaller set of
states, while the valence orbitals are not so constrained.
This provides an effective model for targets with a mostly fixed set of core
electron states, while allowing the valence electrons to interact fully with the
projectile.

\paragraph{Completeness of Target Pseudostates}
\label{sec:th-ccc-target-compl}

As a result of the completeness of the Laguerre basis, the set of target
pseudostates will be separable into a set of bounded pseudostates which will
form an approximation of the true target discrete spectrum, and a set of
unbounded pseudostates which will provide a discretisation of the true continuum
of unbounded states.
Without loss of generality, we order the target pseudostates by increasing
pseudoenergy, $\epsilon_{1}^{\lr{N}} < \dotsc < \epsilon_{N_{T}}^{\lr{N}}$,
which allows us to express the separability of the spectrum in the form
\begin{equation}
  \label{eq:target-spectrum-separable}
  \lrset{\ket*{\Phi_{n}^{\lr{N}}}}_{n = 1}^{N_{T}}
  =
  \lrset{\ket*{\Phi_{n}^{\lr{N}}}}_{n = 1}^{N_{B}}
  \cup
  \lrset{\ket*{\Phi_{n}^{\lr{N}}}}_{n = N_{B} + 1}^{N_{T}}
  ,
\end{equation}
where $\epsilon_{n}^{\lr{N}} < 0$ for $n = 1, \dotsc, N_{B}$, and where
$\epsilon_{n}^{\lr{N}} \geq 0$ for $n = N_{B} + 1, \dotsc, N_{T}$.
Note that $N_{B}$ is the number of bounded pseudostates, and we write
$N_{U} = N_{T} - N_{B}$ to represent the number of unbounded pseudostates, both of
which are dependent on $N$ by consequence of the construction of the target
pseudostates.

The projection operator for the target pseudostates, $\hat{I}_{T}^{\lr{N}}$, is
of the form
\begin{equation}
  \label{eq:target-projection}
  \hat{I}_{T}^{\lr{N}}
  =
  \sum_{n = 1}^{N_{T}}
  \ketbra*
  {\Phi_{n}^{\lr{N}}}
  {\Phi_{n}^{\lr{N}}}
  =
  \sum_{n = 1}^{N_{B}}
  \ketbra*
  {\Phi_{n}^{\lr{N}}}
  {\Phi_{n}^{\lr{N}}}
  +
  \sum_{n = N_{B} + 1}^{N_{T}}
  \ketbra*
  {\Phi_{n}^{\lr{N}}}
  {\Phi_{n}^{\lr{N}}}
  ,
\end{equation}
and so in the limit as $N \to \infty$, the sum over the bounded pseudostates
will converge to the sum over the true target discrete states
and the sum over the unbounded pseudostates will converge to a discretisation of
the integral over the true continuum spectrum.
Whence, it follows that projection operator for the target pseudostates
converges to the identity operator, for $\mathcal{H}_{T}$,
in the limit as $N \to \infty$; that is,
\begin{equation}
  \label{eq:target-projection-convergence}
  \lim_{N \to \infty}
  \hat{I}_{T}^{\lr{N}}
  =
  \hat{I}_{T}
  .
\end{equation}
A more rigorous discussion on the suitabiliy of representing unbounded states in
the Laguerre basis is provided in \cite[5-9]{BRAY19951}.

\subsubsection{Total Wavefunction}
\label{sec:th-ccc-total}

The total wavefunction
$\ket*{\Psi^{\lr{+}}} \in \Lambda^{1 + n_{\mathrm{e}}}\lr{\mathcal{H}}$ is
defined to be an eigenstate of the total Hamiltonian $\hat{H}$ with total
energy $E$ and specified to have outgoing spherical-wave boundary conditions,
\begin{equation}
  \label{eq:total-wavefunction}
  \hat{H}
  \ket*{\Psi^{\lr{+}}}
  =
  E
  \ket*{\Psi^{\lr{+}}}
  ,
\end{equation}
where $\hat{H}$ is of the form
\begin{equation}
  \label{eq:total-hamiltonian}
  \hat{H}
  =
  \hat{H}_{T}
  +
  \hat{K}_{0}
  +
  \hat{V}_{0}
  +
  \sum_{m = 1}^{n_{\mathrm{e}}}
  \hat{V}_{0, m}
  ,
\end{equation}
where $\hat{H}_{T}$ is the target Hamiltonian, defined in
\eqref{eq:target-hamiltonian}, $\hat{K}_{0}$ is the projectile electron
kinetic operator, $\hat{V}_{0}$ is the projectile electron-nuclei potential
operator, and $\hat{V}_{0, m}$ are the projectile electron-target electron
potential operators.
The following treatment of the total wavefunction is of a similar form to
\cite[202-204]{AJP_BRAY1996}.

To ensure that the total wavefunction is anti-symmetric we utilise the
anti-symmetriser, defined in \eqref{eq:anti-symmetriser}, to construct it
explicitly
\begin{equation}
  \label{eq:total-wavefunction-antisymmetrisation}
  \ket*{\Psi^{\lr{+}}}
  =
  \hat{A}
  \ket*{\psi^{\lr{+}}}
  =
  \lrsq[\bigg]
  {
    1
    -
    \sum_{m = 1}^{n_{\mathrm{e}}}
    \hat{P}_{0, m}
  }
  \ket*{\psi^{\lr{+}}}
  ,
\end{equation}
where $\hat{P}_{0, m}$ are the pairwise electron exchange operators defined in
\eqref{eq:pairwise-exchange}, and where
$\ket*{\psi^{\lr{+}}} \in \mathcal{H}_{T} \otimes \mathcal{H}$ is the
unsymmetrised total wavefunction.
As the target states are already anti-symmetric by construction, the
anti-symmetriser has assumed a simpler form - requiring only permutations
of the unsymmetrised projectile state with the spin-orbital states of the
target electrons.
Note that we have omitted the $\lr{1 + n_{\mathrm{e}}}!$ term in $\hat{A}$,
since it is a scalar term which can be normalised away when required.

To construct the unsymmetrised total wavefunction $\ket*{\psi^{\lr{+}}}$ we
perform a multichannel expansion, projecting it onto the target psuedostates,
\begin{equation}
  \label{eq:total-wavefunction-unsymmetrised-n}
  \ket*{\psi^{\lr{N, +}}}
  =
  \hat{I}_{T}^{\lr{N}}
  \ket*{\psi^{\lr{+}}}
  =
  \sum_{n = 1}^{N_{T}}
  \ket*{\Phi_{n}^{\lr{N}}}
  \bra*{\Phi_{n}^{\lr{N}}}
  \ket*{\psi^{\lr{+}}}
  =
  \sum_{n = 1}^{N_{T}}
  \ket*{\Phi_{n}^{\lr{N}} F_{n}^{\lr{N}}}
  ,
\end{equation}
where $\ket*{F_{n}^{\lr{N}}} = \bra*{\Phi_{n}^{\lr{N}}}\ket*{\psi^{\lr{+}}}$ are
the multichannel weight functions, and note that as a result of
\eqref{eq:target-projection-convergence}, that
\begin{equation}
  \label{eq:total-wavefunction-unsymmetrised-convergence}
  \ket*{\psi^{\lr{+}}}
  =
  \lim_{N \to \infty}
  \hat{I}_{T}^{\lr{N}}
  \ket*{\psi^{\lr{+}}}
  =
  \lim_{N \to \infty}
  \ket*{\psi^{\lr{N, +}}}
  .
\end{equation}
Similarly, the total wavefunction constructed from the projection of the
unsymmetrised total wavefunction onto the target pseudostates is written in the
form
\begin{equation}
  \label{eq:total-wavefunction-antisymmetrisation-n}
  \ket*{\Psi^{\lr{N, +}}}
  =
  \hat{A}
  \ket*{\psi^{\lr{N, +}}}
  =
  \lrsq[\bigg]
  {
    1
    -
    \sum_{m = 1}^{n_{\mathrm{e}}}
    \hat{P}_{0, m}
  }
  \ket*{\psi^{\lr{N, +}}}
  ,
\end{equation}
and we note that as a result of \eqref{eq:target-projection-convergence}, that
\begin{equation}
  \label{eq:total-wavefunction-antisymmetrisation-convergence}
  \ket*{\Psi^{\lr{+}}}
  =
  \lim_{N \to \infty}
  \ket*{\Psi^{\lr{N, +}}}
  .
\end{equation}

However, after projecting the unsymmetrised total wavefunction with the
projection operator for the target pseudostates, the multichannel expansion is
not uniquely defined, since for any state
$\ket*{\omega^{\lr{N}}} \in \ker\lr{\hat{A}\hat{I}_{T}^{\lr{N}}}$ and scalar
$\alpha \in \complex$, the multichannel
expansion of $\ket*{\psi^{\lr{N, +}}} + \alpha \ket*{\omega^{\lr{N}}}$ will be
identical to that of $\ket*{\psi^{\lr{N, +}}}$.
To resolve this dilemma, we first note that the multichannel weight functions
$\ket*{F_{n}^{\lr{N}}}$ are within the span of the one-electron spin orbitals
$\lrset{\ket*{\chi_{i}^{\lr{N}}}}_{i = 1}^{2N}$, used to construct the Slater
determinants, \eqref{eq:target-slater-determinant}, with which the target
states are expanded.
Hence, we impose the constraint that for any of the one-electron spin orbitals
$\ket*{\chi_{i}^{\lr{N}}}$, that
\begin{equation}
  \label{eq:multichannel-constraint}
  \hat{P}_{0, m}
  \ket*{\Phi_{n}^{\lr{N}} \chi_{i}^{\lr{N}}}
  =
  -
  \ket*{\Phi_{n}^{\lr{N}} \chi_{i}^{\lr{N}}}
  .
\end{equation}
which can be seen as an explicit imposition of \eqref{eq:pairwise-exchange}.
With this constraint in place, it can then be shown that
$\dim\ker\lr{\hat{A}\hat{I}_{T}^{\lr{N}}} = 0$, whence it follows that the
multichannel expansion of $\ket*{\psi^{\lr{N, +}}}$ is now unique in determining
$\ket*{\Psi^{\lr{N, +}}}$.
% To make these constraints explicit, we note as a consequence we have
% \begin{equation}
%   \label{eq:multichannel-constraint-2}
%   \hat{P}_{0, m}
%   \ket*{\psi^{\lr{N, +}}}
%   =
%   -
%   \ket*{\psi^{\lr{N, +}}}
%   ,
%   \qq{for}
%   m = 1, \dotsc, n_{\mathrm{e}}
%   .
% \end{equation}

\subsubsection{Convergent Close-Coupling Equations}
\label{sec:th-ccc-eq}

We present a derivation for the Convergent Close-Coupling (CCC) equations,
beginning with the Schr\"odinger equation for the total wavefunction
$\ket*{\Psi^{\lr{+}}}$ presented in \eqref{eq:total-wavefunction}.
This shall be re-arranged to yield the Lippmann-Schwinger equation, which will
then be solved using the CCC formalism to obtain the matrix elements of the
$\hat{T}$ operator - with which scattering statistics can be calculated.

\paragraph{Lippmann-Schwinger Equation}
\label{sec:th-ccc-eq-ls}

We consider an eigenstate $\ket*{\Psi}$ of a Hamiltonian $\hat{H}$, with
eigenenergy $E$, for which the Schr\"odinger equation is of the form
\begin{equation}
  \label{eq:schrodinger}
  \hat{H}
  \ket{\Psi}
  =
  \hat{H}_{\rm{A}}
  \ket{\Psi}
  +
  \hat{V}
  \ket{\Psi}
  =
  E
  \ket{\Psi}
  ,
\end{equation}
where $\hat{H}_{\rm{A}}$ is the unbounded asymptotic Hamiltonian and $\hat{V}$
is a potential.
This expression can be rearranged to the form
\begin{equation}
  \label{eq:schrodinger-2}
  \lrsq
  {
    E
    -
    \hat{H}_{\rm{A}}
  }
  \ket{\Psi}
  =
  \hat{V}
  \ket{\Psi}
  .
\end{equation}
Suppose that $\lrset{\ket*{\Omega_{\alpha}}}$ are the (countably and
uncountably infinite) eigenstates of the asymptotic Hamiltonian, with
corresponding eigenvalues
$\varepsilon_{\alpha}$,
\begin{equation}
  \label{eq:schrodinger-asymptotic}
  \hat{H}_{\rm{A}}
  \ket{\Omega_{\alpha}}
  =
  \varepsilon_{\alpha}
  \ket{\Omega_{\alpha}}
  .
\end{equation}
We note that where $\varepsilon_{\alpha} = E$, it follows that
$\ket*{\Omega_{\alpha}} \in \ker\lr{E - \hat{H}_{\rm{A}}}$; for a given energy
$E$, we denote these particular asymptotic states by
$\ket*{\Omega_{\alpha}^{\lr{E}}}$ and say that they are on-shell states, and
that the energies of these states are on-shell.
We now define the Green's operator $\hat{G}_{\lr{E}}$, to be such that
\begin{equation}
  \label{eq:greens-operator}
  \hat{G}_{\lr{E}}
  \lrsq
  {
    E
    -
    \hat{H}_{\rm{A}}
  }
  =
  \hat{I}
  =
  \lrsq
  {
    E
    -
    \hat{H}_{\rm{A}}
  }
  \hat{G}_{\lr{E}}
  ,
\end{equation}
whence we obtain a general form of the Lippmann-Schwinger equation,
\begin{equation}
  \label{eq:ls}
  \ket{\Psi}
  =
  \sum_{\alpha : \varepsilon_{\alpha} = E}\int
  C_{\alpha}
  \ket*{\Omega_{\alpha}^{\lr{E}}}
  +
  \hat{G}_{\lr{E}}
  \hat{V}
  \ket{\Psi}
  ,
\end{equation}
where $C_{\alpha}$ are arbitrary scalar coefficients.
We note that in this context, the sum taken over the indexes of the asymptotic
eigenstates represents a sum over the countably infinite states, and an
integration over the uncountably infinite states, for which the eigenenergy
$\varepsilon_{\alpha}$ is equal to $E$.
The inclusion of the selected asymptotic eigenstates is required as they are in
the kernel of $\lrsq{E - \hat{H}_{\rm{A}}}$, thus forming the homogenous
solutions to the Lippmann-Schwinger equation.
This can be demonstrated by applying the operator $\lrsq{E - \hat{H}_{\rm{A}}}$
on the left of \eqref{eq:ls},
\begin{alignat*}{2}
  % \label{eq:ls-kernel}
  &
  \lrsq
  {
    E
    -
    \hat{H}_{\rm{A}}
  }
  \ket{\Psi}
  &
  {}={}
  &
  \sum_{\alpha : \varepsilon_{\alpha} = E}\int
  C_{\alpha}
  \lrsq
  {
    E
    -
    \hat{H}_{\rm{A}}
  }
  \ket*{\Omega_{\alpha}^{\lr{E}}}
  +
  \lrsq
  {
    E
    -
    \hat{H}_{\rm{A}}
  }
  \hat{G}_{\lr{E}}
  \hat{V}
  \ket{\Psi}
  \\
  &
  &
  {}={}
  &
  \sum_{\alpha : \varepsilon_{\alpha} = E}\int
  C_{\alpha}
  \ket{0}
  +
  \hat{I}
  \hat{V}
  \ket{\Psi}
  \\
  &
  &
  {}={}
  &
  \hat{V}
  \ket{\Psi}
  .
\end{alignat*}
At this point, we note that selecting the values of the coefficients
$C_{\alpha}$ amounts to specifying a boundary condition for the eigenstate
$\ket{\Psi}$.
By consequence of the linearity of \eqref{eq:ls}, we may therefore simplify
the generalised sum/integral, without loss of generality, by considering
eigenstates of the form
\begin{equation}
  \label{eq:ls-boundary}
  \ket{\Psi_{\alpha}}
  =
  \ket*{\Omega_{\alpha}^{\lr{E}}}
  +
  \hat{G}_{\lr{E}}
  \hat{V}
  \ket{\Psi_{\alpha}}
  ,
\end{equation}
for a particular
$\ket*{\Omega_{\alpha}^{\lr{E}}} \in \ker\lr{E - \hat{H}_{\rm{A}}}$, and we say
that $\ket{\Psi_{\alpha}}$ is the eigenstate of $\hat{H}$ corresponding to the
boundary condition specified by the asymptotic eigenstate
$\ket*{\Omega_{\alpha}^{\lr{E}}}$.
We now define the $\hat{T}$ operator to be such that
\begin{equation}
  \label{eq:t-operator}
  \ket{\Psi_{\alpha}}
  =
  \lrsq
  {
    \hat{I}
    +
    \hat{G}_{\lr{E}}
    \hat{T}
  }
  \ket*{\Omega_{\alpha}^{\lr{E}}}
  ,
\end{equation}
which is equivalently defined by writing
\begin{equation}
  \label{eq:t-operator-v}
  \hat{T}
  \ket*{\Omega_{\alpha}^{\lr{E}}}
  =
  \hat{V}
  \ket{\Psi_{\alpha}}
  .
\end{equation}
Furthermore, we have that
\begin{alignat*}{4}
  &
  \ket{\Psi_{\alpha}}
  &&
  {}={}
  &&
  \ket*{\Omega_{\alpha}^{\lr{E}}}
  +
  \hat{G}_{\lr{E}}
  \hat{V}
  \ket{\Psi_{\alpha}}
  &
  \\
  &
  &&
  {}={}
  &&
  \ket*{\Omega_{\alpha}^{\lr{E}}}
  +
  \hat{G}_{\lr{E}}
  \hat{V}
  \lrsq[\big]
  {
    \hat{I}
    +
    \hat{G}_{\lr{E}}
    \hat{T}
  }
  \ket*{\Omega_{\alpha}^{\lr{E}}}
  &
  \\
  &
  &&
  {}={}
  &&
  \lrsq[\big]
  {
    \hat{I}
    +
    \hat{G}_{\lr{E}}
    \hat{V}
    +
    \hat{G}_{\lr{E}}
    \hat{V}
    \hat{G}_{\lr{E}}
    \hat{T}
  }
  \ket*{\Omega_{\alpha}^{\lr{E}}}
  &
  \\
  &
  &&
  {}={}
  &&
  \lrsq[\big]
  {
    \hat{I}
    +
    \hat{G}_{\lr{E}}
    \lr
    {
      \hat{V}
      +
      \hat{V}
      \hat{G}_{\lr{E}}
      \hat{T}
    }
  }
  \ket*{\Omega_{\alpha}^{\lr{E}}}
  ,
  &
\end{alignat*}
whence it follows that $\hat{T}$ can be written in the form
\begin{equation}
  \label{eq:ls-t-greens}
  \hat{T}
  \ket*{\Omega_{\alpha}^{\lr{E}}}
  =
  \lrsq
  {
    \hat{V}
    +
    \hat{V}
    \hat{G}_{\lr{E}}
    \hat{T}
  }
  \ket*{\Omega_{\alpha}^{\lr{E}}}
  ,
\end{equation}
yielding the formulation of the Lippmann-Schwinger equation in terms of the
$\hat{T}$ operator.
At this point we consider the explicit form of the Green's operator
$\hat{G}_{\lr{E}}$.
First, we note that the asymptotic eigenstates are complete in the sense that
they provide a resolution of the identity
\begin{equation}
  \label{eq:asymptotic-complete}
  \hat{I}
  =
  \sum_{\gamma}\int
  \ketbra*{\Omega_{\gamma}}{\Omega_{\gamma}}
  ,
\end{equation}
and a spectral decomposition of the asymptotic Hamiltonian
\begin{equation*}
  \hat{H}_{\rm{A}}
  =
  \sum_{\gamma}\int
  \varepsilon_{\gamma}
  \ketbra*{\Omega_{\gamma}}{\Omega_{\gamma}}
  .
\end{equation*}
It therefore follows from the definition of the Green's operator,
\eqref{eq:greens-operator}, that we must have
\begin{alignat*}{2}
  &
  &
  \hat{G}_{\lr{E}}
  \lrsq
  {
    E
    -
    \hat{H}_{\rm{A}}
  }
  {}={}
  &
  \hat{I}
  \\
  &
  &
  \sum_{\gamma}\int
  \lr
  {
    E
    -
    \varepsilon_{\gamma}
  }
  \hat{G}_{\lr{E}}
  \ketbra*{\Omega_{\gamma}}{\Omega_{\gamma}}
  {}={}
  &
  \sum_{\gamma}\int
  \ketbra*{\Omega_{\gamma}}{\Omega_{\gamma}}
  ,
\end{alignat*}
whence it follows that the spectral decomposition of the Green's operator is of
the form
\begin{equation}
  \label{eq:greens-operator-spectral}
  \hat{G}_{\lr{E}}
  =
  \sum_{\gamma}\int
  \dfrac
  {
    \ketbra*{\Omega_{\gamma}}{\Omega_{\gamma}}
  }
  {
    E
    -
    \varepsilon_{\gamma}
  }
  .
\end{equation}
However, this expression is not well-defined, as it is singular for the
asymptotic states $\ket*{\Omega_{\gamma}^{\lr{E}}}$ for which
$\varepsilon_{\gamma} = E$.
This problem can be overcome by regularising the Green's operator to either the
incoming $\hat{G}_{\lr{E, -}}$ or outgoing $\hat{G}_{\lr{E, +}}$ forms,
\begin{equation}
  \label{eq:greens-operator-regularise}
  \hat{G}_{\lr{E, \pm}}
  =
  \lim_{\eta \to 0}
  \sum_{\gamma}\int
  \dfrac
  {
    \ketbra*{\Omega_{\gamma}}{\Omega_{\gamma}}
  }
  {
    E - \varepsilon_{\gamma} \pm \imath\eta
  }
  =
  \sum_{\gamma}\int
  \dfrac
  {
    \ketbra*{\Omega_{\gamma}}{\Omega_{\gamma}}
  }
  {
    E - \varepsilon_{\gamma} \pm \imath 0
  }
  ,
\end{equation}
where the presence of the imaginary limit ensures that the integral is
well-defined for all $\varepsilon_{\gamma}$.
We elect to use the outgoing Green's operator $\hat{G}_{\lr{E, +}}$ as we are
concerned with the outgoing behaviour of the eigenstate
$\ket*{\Psi_{\alpha}^{\lr{+}}}$.
We can now re-write the Lippmann-Schwinger equation in the following form
\begin{equation}
  \label{eq:ls-t}
  \mel*
  {\Omega_{\alpha}}
  {\hat{T}}
  {\Omega_{\beta}^{\lr{E}}}
  =
  \mel*
  {\Omega_{\alpha}}
  {\hat{V} }
  {\Omega_{\beta}^{\lr{E}}}
  +
  \sum_{\gamma}\int
  \dfrac
  {
    \mel*
    {\Omega_{\alpha}}
    {\hat{V}}
    {\Omega_{\gamma}}
    \mel*
    {\Omega_{\gamma}}
    {\hat{T}}
    {\Omega_{\beta}^{\lr{E}}}
  }
  {
    E - \varepsilon_{\gamma} + \imath 0
  }
  ,
\end{equation}
which expresses the representation of the operator $\hat{T}$, for a given energy
$E$, in terms of the asymptotic eigenstates $\lrset{\ket*{\Omega_{\alpha}}}$ and
the on-shell asymptotic eigenstates $\lrset{\ket*{\Omega_{\beta}^{\lr{E}}}}$.

\paragraph{Convergent Close-Coupling Formalism}
\label{sec:th-ccc-eq-form}

In the Convergent Close-Coupling formalism, the Lippmann-Schwinger equation in
terms of the $\hat{T}$ operator, \eqref{eq:ls-t}, is solved in momentum space.
We preface this discussion with a minor note, that the notation for the
asymptotic Hamiltonian $\hat{H}_{\rm{A}}$ is to be distinguished from the
notation for the anti-symmetriser $\hat{A}$, and is unrelated.

We split the Hamiltonian, from \eqref{eq:total-hamiltonian}, into an
asymptotic Hamiltonian and a potential in the form
\begin{equation}
  \label{eq:ccc-hamiltonian}
  \hat{H}
  =
  \hat{H}_{T}
  +
  \hat{K}_{0}
  +
  \hat{V}_{0}
  +
  \sum_{m = 1}^{n_{\mathrm{e}}}
  \hat{V}_{0, m}
  =
  \hat{H}_{\rm{A}}
  +
  \hat{W}
  ,
\end{equation}
where the asymptotic Hamiltonian is of the form
\begin{equation}
  \label{eq:ccc-asymptotic-hamiltonian}
  \hat{H}_{\rm{A}}
  =
  \hat{H}_{T}
  +
  \hat{K}_{0}
  +
  \hat{U}_{0}
  ,
\end{equation}
and where the potential, modelling the interaction between the projectile and
target states, is of the form
\begin{equation}
  \label{eq:ccc-potential}
  \hat{W}
  =
  \hat{V}_{0}
  +
  \sum_{m = 1}^{n_{\mathrm{e}}}
  \hat{V}_{0, m}
  -
  \hat{U}_{0}
  ,
\end{equation}
where $\hat{U}_{0}$ is an asymptotic potential acting on the projectile, which
can be chosen arbitrarily.
A suitable choice for this potential is that of a Coulomb potential with a
charge corresponding to the asymptotic charge of the target system, whence
$\braket*{\vb{r}}{\hat{W}} = W\lr{r, \Omega} \to 0$ as $r \to \infty$.
Such a selection for $\hat{U}_{0}$ adapts the projectile states to the target
system, without loss of generality, and can lead to improvement in computational
performance, as discussed in \cite[204]{AJP_BRAY1996}.

The asymptotic eigenstates are therefore taken to be of the form
\begin{equation}
  \label{eq:ccc-asymptotic-states}
  \ket*{\Omega_{\alpha}}
  =
  \ket*{\Phi_{\alpha} \vb{k}_{\alpha}}
  \approx
  \ket*{\Phi_{n_{\alpha}}^{\lr{N}} \vb{k}_{\alpha}}
  ,
\end{equation}
where $\lrset{\ket*{\Phi_{n}^{\lr{N}}}}_{i = 1}^{N_{T}}$ are the target
pseudostates, defined in \eqref{eq:target-states}, which satisfy
\begin{equation}
  \label{eq:ccc-target}
  \mel*
  {\Phi_{i}^{\lr{N}}}
  {\hat{H}_{T}}
  {\Phi_{j}^{\lr{N}}}
  =
  \epsilon_{i}^{\lr{N}}
  \delta_{i, j}
  ,
\end{equation}
and where $\ket*{\vb{k}_{\alpha}}$ are the continuum waves (which could be
plane, distorted, or Coulomb waves depending on the choice of $\hat{U}_{0}$),
defined to be eigenstates of the projectile component of the asymptotic
Hamiltonian,
\begin{equation}
  \label{eq:ccc-waves}
  \lrsq
  {
    \hat{K}_{0}
    +
    \hat{U}_{0}
  }
  \ket*{\vb{k}_{\alpha}}
  =
  \tfrac{1}{2}
  k_{\alpha}^{2}
  \ket*{\vb{k}_{\alpha}}
  ,
\end{equation}
whence it can be seen that the asymptotic eigenenergies are of the form
\begin{equation}
  \label{eq:ccc-asymptotic-energy}
  \varepsilon_{\alpha}
  =
  \epsilon_{n_{\alpha}}
  +
  \tfrac{1}{2}
  k_{\alpha}^{2}
  \approx
  \epsilon_{n_{\alpha}}^{\lr{N}}
  +
  \tfrac{1}{2}
  k_{\alpha}^{2}
  .
\end{equation}
Furthermore, the total wavefunction is taken to be of the form
\begin{equation}
  \label{eq:ccc-total-wavefunction}
  \ket*{\Psi_{\alpha}^{\lr{+}}}
  =
  \hat{A}
  \ket*{\psi_{\alpha}^{\lr{+}}}
  \approx
  \hat{A}
  \hat{I}_{T}^{\lr{N}}
  \ket*{\psi_{\alpha}^{\lr{+}}}
  =
  \hat{A}
  \ket*{\psi_{\alpha}^{\lr{N, +}}}
  =
  \ket*{\Psi_{\alpha}^{\lr{N, +}}}
  ,
\end{equation}
as in \eqref{eq:total-wavefunction-antisymmetrisation}, where $\hat{A}$ is the
anti-symmetriser operator, defined in \eqref{eq:anti-symmetriser}, and is
subject to the constraints imposed in \eqref{eq:multichannel-constraint} to
ensure uniqueness.
We note that with these expressions for the asymptotic eigenstates and the total
wavefunction, that the $\hat{T}$ operator is related to the potential $\hat{W}$
by the expression
\begin{equation}
  \label{eq:ccc-t-w}
  \hat{T}
  \ket*{\Phi_{n_{\alpha}}^{\lr{N}} \vb{k}_{\alpha}}
  =
  \hat{W}
  \ket*{\Psi_{\alpha}^{\lr{N, +}}}
  =
  \hat{W}
  \hat{A}
  \hat{I}_{T}^{\lr{N}}
  \ket*{\psi_{\alpha}^{\lr{+}}}
  =
  \hat{W}
  \hat{A}
  \ket*{\psi_{\alpha}^{\lr{N, +}}}
  .
\end{equation}
However, it is possible to recast the potential $\hat{W}$ in a form $\hat{V}$
which accounts for the explicit anti-symmetrisation of the total wavefunction;
that is, which allows us to write the CCC equations without direct reference to
the anti-symmetriser $\hat{A}$.
To do this, we first note that
\begin{equation*}
  0
  =
  \lrsq
  {
    E
    -
    \hat{H}
  }
  \ket*{\Psi_{\alpha}^{\lr{+}}}
  =
  \lrsq
  {
    E
    -
    \hat{H}
  }
  \hat{A}
  \ket*{\psi_{\alpha}^{\lr{+}}}
  ,
\end{equation*}
with the operator on the right hand side expanding to the form
\begin{equation*}
  \lrsq
  {
    E
    -
    \hat{H}
  }
  \hat{A}
  =
  \lrsq[\bigg]
  {
    E
    -
    \hat{H}
    -
    \lrsq
    {
      E
      -
      \hat{H}
    }
    \sum_{m = 1}^{n_{\mathrm{e}}}
    \hat{P}_{0, m}
  }
  =
  \lrsq[\bigg]
  {
    E
    -
    \hat{H}_{\rm{A}}
    -
    \hat{W}
    -
    \lrsq
    {
      E
      -
      \hat{H}
    }
    \sum_{m = 1}^{n_{\mathrm{e}}}
    \hat{P}_{0, m}
  }
  ,
\end{equation*}
where again we make sure to distinguish the notation for the asymptotic
Hamiltonian $\hat{H}_{\rm{A}}$ and the anti-symmetriser $\hat{A}$.
We therefore define the explicitly anti-symmetrised potential $\hat{V}$ to be of
the form
\begin{equation}
  \label{eq:ccc-v}
  \hat{V}
  =
  \hat{W}
  +
  \lrsq
  {
    E
    -
    \hat{H}
  }
  \sum_{m = 1}^{n_{\mathrm{e}}}
  \hat{P}_{0, m}
  =
  \hat{V}_{0}
  +
  \sum_{m = 1}^{n_{\mathrm{e}}}
  \hat{V}_{0, m}
  -
  \hat{U}_{0}
  +
  \lrsq
  {
    E
    -
    \hat{H}
  }
  \sum_{m = 1}^{n_{\mathrm{e}}}
  \hat{P}_{0, m}
  ,
\end{equation}
for which we can see that
\begin{equation*}
  \label{eq:ccc-v-2}
  0
  =
  \lrsq
  {
    E
    -
    \hat{H}
  }
  \hat{A}
  \ket*{\psi_{\alpha}^{\lr{+}}}
  =
  \lrsq
  {
    E
    -
    \lrsq
    {
      \hat{H}_{\rm{A}}
      +
      \hat{V}
    }
  }
  \ket*{\psi_{\alpha}^{\lr{+}}}
  ,
\end{equation*}
which is to say that the Lippmann-Schwinger equation \eqref{eq:ls-t} can be
written in terms of the unsymmetrised total wavefunction
$\ket*{\psi_{\alpha}^{\lr{+}}}$, rather than the anti-symmetric total
wavefunction $\ket*{\Psi_{\alpha}^{\lr{+}}}$.
Specifically, this allows us to write the $\hat{T}$ operator in the form
\begin{equation}
  \label{eq:ccc-t-v}
  \hat{T}
  \ket*{\Phi_{n_{\alpha}}^{\lr{N}} \vb{k}_{\alpha}}
  =
  \hat{V}
  \hat{I}_{T}^{\lr{N}}
  \ket*{\psi_{\alpha}^{\lr{+}}}
  =
  \hat{V}
  \ket*{\psi_{\alpha}^{\lr{N, +}}}
  .
\end{equation}
We then have the Convergent Close-Coupling equations in terms of the $\hat{T}$
operator
\begin{alignat}{2}
  \label{eq:ccc-ls-t}
  &
  \mel*
  {\vb{k}_{f} \Phi_{n_{f}}^{\lr{N}}}
  {\hat{T}}
  {\Phi_{n_{i}}^{\lr{N}} \vb{k}_{i}}
  &
  {}={}
  &
  \mel*
  {\vb{k}_{f} \Phi_{n_{f}}^{\lr{N}}}
  {\hat{V}}
  {\Phi_{n_{i}}^{\lr{N}} \vb{k}_{i}}
  \\
  &
  &
  {}+{}
  &
  \sum_{n = 1}^{N_{T}}
  \int
  \dd{\vb{k}}
  \dfrac
  {
    \mel*
    {\vb{k}_{f} \Phi_{n_{f}}^{\lr{N}}}
    {\hat{V}}
    {\Phi_{n}^{\lr{N}} \vb{k}}
    \mel*
    {\vb{k} \Phi_{n}^{\lr{N}}}
    {\hat{T}}
    {\Phi_{n_{i}}^{\lr{N}} \vb{k}_{i}}
  }
  {
    E
    -
    \epsilon_{n}^{\lr{N}}
    -
    \tfrac{1}{2}
    k^{2}
    \pm \imath 0
  }
  \nonumber
  ,
\end{alignat}
forming a set of $\complex$-valued matrix equations  which are numerically
solved to yield the $T$ matrix, from which information about the total
wavefunction $\ket*{\Psi_{i}^{\lr{N, +}}}$ can be derived.
However, it is possible to re-write the Convergent Close-Coupling equations in
terms of an operator $\hat{K}$,
\begin{alignat}{2}
  \label{eq:ccc-ls-k}
  &
  \mel*
  {\vb{k}_{f} \Phi_{n_{f}}^{\lr{N}}}
  {\hat{K}}
  {\Phi_{n_{i}}^{\lr{N}} \vb{k}_{i}}
  &
  {}={}
  &
  \mel*
  {\vb{k}_{f} \Phi_{n_{f}}^{\lr{N}}}
  {\hat{V}}
  {\Phi_{n_{i}}^{\lr{N}} \vb{k}_{i}}
  \\
  &
  &
  {}+{}
  &
  \sum_{n = 1}^{N_{T}}
  \mathcal{P}
  \int
  \dd{\vb{k}}
  \dfrac
  {
    \mel*
    {\vb{k}_{f} \Phi_{n_{f}}^{\lr{N}}}
    {\hat{V}}
    {\Phi_{n}^{\lr{N}} \vb{k}}
    \mel*
    {\vb{k} \Phi_{n}^{\lr{N}}}
    {\hat{K}}
    {\Phi_{n_{i}}^{\lr{N}} \vb{k}_{i}}
  }
  {
    E
    -
    \epsilon_{n}^{\lr{N}}
    -
    \tfrac{1}{2}
    k^{2}
  }
  \nonumber
  ,
\end{alignat}
where $\mathcal{P}$ indicates that the principal value of the integral is taken,
which forms a set of $\real$-valued matrix equations which can be solved more
efficiently, to yield the $K$ matrix.
The $T$ matrix can then be reconstructed from the $K$ matrix by the identity
\cite[9]{BRAY19951}
\begin{equation}
  \label{eq:ccc-t-k}
  \mel*
  {\vb{k}_{f} \Phi_{n_{f}}^{\lr{N}}}
  {\hat{K}}
  {\Phi_{n_{i}}^{\lr{N}} \vb{k}_{i}}
  =
  \sum_{n = 1}^{N_{T}}
  \mel*
  {\vb{k}_{f} \Phi_{n_{f}}^{\lr{N}}}
  {\hat{T}}
  {\Phi_{n}^{\lr{N}} \vb{k}_{n}}
  \lr[\big]
  {
    \delta_{n, i}
    +
    \imath
    \pi
    k_{n}
    \mel*
    {\vb{k}_{n} \Phi_{n}^{\lr{N}}}
    {\hat{K}}
    {\Phi_{n_{i}}^{\lr{N}} \vb{k}_{i}}
  }
  ,
\end{equation}
where $k_{n}$ are the on-shell projectile momenta which satisfy
\begin{equation}
  \label{eq:ccc-on-shell-momentum}
  E
  =
  \epsilon_{n}^{\lr{N}}
  +
  \tfrac{1}{2}
  k_{n}^{2}
  \qq{for}
  n = 1, \dotsc, N_{T}.
\end{equation}

We note that the matrix equations \eqref{eq:ccc-ls-t}, as well as
\eqref{eq:ccc-ls-k} and \eqref{eq:ccc-t-k}, are computationally
parameterised by the set of target pseudostates
$\lrset{\ket*{\Phi_{n}^{\lr{N}}}}_{n = 1}^{N_{T}}$ and the discretisation of
the projectile spectrum.
In turn, the target pseudostates are parameterised by the number of Slater
determinants $N_{T}$ used in their construction, and the number of basis
functions $N$ used to construct the one-electron orbitals from which the Slater
determinants are built.
Furthermore, we note that matrix equations in this form do not explicitly
include the constraints, detailed in \eqref{eq:multichannel-constraint}, which
guarantee the uniqueness of the explicitly anti-symmetrised multichannel
expansion.

\subsection{Scattering Statistics}
\label{sec:th-ccc-stat}

At this point, we shall make explicit use of the S-wave model, wherein all
partial wave expansions are limited to the $l = 0$ terms;
this has the effect of restricting our attention to asymptotic eigenstates
$\ket*{\Phi_{n}^{\lr{N}} \vb{k}}$ for which the target pseudostate has $l = 0$.
This allows for a simpler presentation of the theory, and a significant
reduction in computational complexity.
Furthermore, calculations performed in the S-wave model are sufficient for the
emergence of scattering phenomena with which we are interested.
Much of the following treatment is generalisable to the inclusion of arbitrary
angular momentum.

Lastly, we note that many of the following statistics can be constructed for a
particular symmetry of the system which is conserved by the scattering process;
examples include total spin and angular momentum.
We shall refrain from specifying the forms of these statistics for specific
symmetries, in lieu of providing a clearer, more general treatment.

\subsubsection{Scattering Amplitudes}
\label{sec:th-ccc-amp}

Once calculated, the matrix elements of the $\hat{T}$ operator yield the
transition amplitudes between asymptotic states, which can then be used to
calculate the scattering amplitudes.
In general terms, the scattering amplitudes can be written in the form
\begin{equation}
  \label{eq:scattering-amplitude}
  f_{\alpha, \beta}
  =
  f_{\alpha, \beta}\lr{\vb{k}_{\alpha}, \vb{k}_{\beta}}
  =
  \mel*
  {\vb{k}_{\alpha} \Phi_{\alpha}}
  {\hat{V}}
  {\Psi_{\beta}}
  =
  \mel*
  {\vb{k}_{\alpha} \Phi_{\alpha}}
  {\hat{T}}
  {\Phi_{\beta} \vb{k}_{\beta}}
  ,
\end{equation}
where the target state $\ket*{\Phi_{\alpha}}$ can be a bounded discrete
state or an unbounded continuum state, corresponding to either an elastic
scattering / a discrete excition transition, or an ionisation transition.
For discrete excitations, the numerically calculated scattering amplitude is
simply of the form
\begin{equation}
  \label{eq:scattering-amplitude-discrete}
  f_{f, i}^{\lr{N}}
  =
  f_{n_{f}, n_{i}}^{\lr{N}}\lr{\vb{k}_{f}, \vb{k}_{i}}
  =
  \mel*
  {\vb{k}_{f} \Phi_{n_{f}}^{\lr{N}}}
  {\hat{T}}
  {\Phi_{n_{i}}^{\lr{N}} \vb{k}_{i}}
  ,
\end{equation}
for on-shell transitions,
\begin{equation}
  \label{eq:on-shell-energy-discrete}
  \epsilon_{n_{f}}^{\lr{N}}
  +
  \tfrac{1}{2}
  k_{f}^{2}
  =
  E
  =
  \epsilon_{n_{i}}^{\lr{N}}
  +
  \tfrac{1}{2}
  k_{i}^{2}
  ,
\end{equation}
with elastic scattering occuring in the case where $n_{f} = n_{i}$,
\begin{equation}
  \label{eq:scattering-amplitude-elastic}
  f_{i}^{\lr{N}}\lr{\vb{k}_{f}, \vb{k}_{i}}
  =
  f_{n_{i}, n_{i}}^{\lr{N}}\lr{\vb{k}_{f}, \vb{k}_{i}}
  =
  \mel*
  {\vb{k}_{f} \Phi_{n_{i}}^{\lr{N}}}
  {\hat{T}}
  {\Phi_{n_{i}}^{\lr{N}} \vb{k}_{i}}
  .
\end{equation}
However, the numerically calculated scattering amplitudes for ionisations,
hereby referred to as ionisation amplitudes,
require a more carefully considered treatment - which we present in a form
similar to that described in \cite{PhysRevLett.89.273201, PhysRevA.90.022710}.
We shall restrict our attention to the case of single ionisation, but leave open
the consideration of ionisation with excitation.
The ionised asymptotic state $\ket*{\Phi_{\alpha} \vb{k}_{\alpha}}$ corresponds
to the breakup of the target state $\ket*{\Phi_{\alpha}}$ into a singly-ionised
target state $\ket*{\Phi_{n_{\alpha}}^{+}}$ (which may be excited) and an ionised
electron in the form of a Coulomb wave $\ket*{\vb{q}_{\alpha}}$; that is,
\begin{equation}
  \label{eq:asymptotic-ionised}
  \ket*{\Phi_{\alpha} \vb{k}_{\alpha}}
  =
  \ket*{\Phi_{n_{\alpha}}^{+} \vb{q}_{\alpha} \vb{k}_{\alpha}}
  ,
\end{equation}
where the energy of the ionised asymptotic state is of the form
\begin{equation}
  \label{eq:asymptotic-ionised-energy}
  E
  =
  \epsilon_{\alpha}
  +
  \tfrac{1}{2}
  k_{\alpha}^{2}
  =
  \epsilon_{n_{\alpha}}^{+}
  +
  \tfrac{1}{2}
  q_{\alpha}^{2}
  +
  \tfrac{1}{2}
  k_{\alpha}^{2}
\end{equation}
where $\epsilon_{n_{\alpha}}^{+}$ is the energy of the singly-ionised target
state, and where $\tfrac{1}{2} q_{\alpha}^{2}$ is the energy of the Coulomb
wave.
It is important to note that in this formulation, the asymptotic state
$\ket*{\Phi_{\alpha} \vb{k}_{\alpha}}$ separates into the asymptotic projectile
state $\ket*{\vb{k}_{\alpha}}$ and the asymptotic target state
$\ket*{\Phi_{\alpha}} = \ket*{\Phi_{n_{\alpha}}^{+} \vb{q}_{\alpha}}$, within
which the Coulomb wave $\ket{\vb{q}_{\alpha}}$ is modelled - thus excluding from
consideration a three-body boundary condition.
This presents an issue however as Coulomb waves are not bounded states, and
thus their coordinate-space representations are not elements of
$L^{2}\lr{\real^{3}}$.
This is the space wherein the coordinate-space representations of the
one-electron states, comprising the target pseudostates, are spanned in terms of
the Laguerre basis, \eqref{eq:laguerre-basis}.
However it can be shown, as discussed in \cite{BRAY19951}, that while the
projection of a continuum wave onto a $N$-dimensional Laguerre basis is only
conditionally convergent as $N$ increases, it is numerically stable.
Hence, the numerically calculated ionisation amplitudes can be written in the
form
\begin{alignat}{2}
  \label{eq:scattering-amplitude-ionisation}
  &
  f_{\alpha, i}^{\lr{N}}
  =
  f_{n_{\alpha}, n_{i}}^{\lr{N}}\lr
  {
    \vb{k}_{\alpha}, \vb{q}_{\alpha}, \vb{k}_{i}
  }
  &
  {}={}
  &
  \mel*
  {\vb{k}_{\alpha} \vb{q}_{\alpha} \Phi_{n_{\alpha}}^{+}}
  {\hat{I}_{T}^{\lr{N}} \hat{T}}
  {\Phi_{n_{i}}^{\lr{N}} \vb{k}_{i}}
  \nonumber
  \\
  &
  &
  {}={}
  &
  \sum_{n = 1}^{N_{T}}
  \braket*
  {\vb{k}_{\alpha} \vb{q}_{\alpha} \Phi_{n_{\alpha}}^{+}}
  {\Phi_{n}^{\lr{N}}}
  \mel*
  {\Phi_{n}^{\lr{N}}}
  {\hat{T}}
  {\Phi_{n_{i}}^{\lr{N}} \vb{k}_{i}}
  \nonumber
  \\
  &
  &
  {}={}
  &
  \sum_{n = 1}^{N_{T}}
  \braket*
  {\vb{q}_{\alpha} \Phi_{n_{\alpha}}^{+}}
  {\Phi_{n}^{\lr{N}}}
  \mel*
  {\vb{k}_{\alpha} \Phi_{n}^{\lr{N}}}
  {\hat{T}}
  {\Phi_{n_{i}}^{\lr{N}} \vb{k}_{i}}
  .
\end{alignat}
However, this expression is problematic as it involves a summation over
not necessarily on-shell terms $\bra*{\vb{k}_{\alpha}\Phi_{n}^{\lr{N}}}$.
If we restrict our attention to only evaluating the ionisation amplitudes
$f_{\alpha, i}^{\lr{N}}$ for ionised asymptotic states
$\ket*{\Phi_{n_{\alpha}}^{+} \vb{q}_{\alpha} \vb{k}_{\alpha}}$ for which the
ionised target energy satisfies
\begin{equation}
  \label{eq:asymptotic-ionisation-energy-constraint}
  \epsilon_{\alpha}
  =
  \epsilon_{n_{\alpha}}^{+}
  +
  \tfrac{1}{2}
  q_{\alpha}^{2}
  =
  \epsilon_{n_{\alpha}}^{\lr{N}}
  ,
\end{equation}
for one of the target pseudoenergies $\epsilon_{n_{\alpha}}^{\lr{N}}$,
corresponding to the target pseudostate $\ket*{\Phi_{n_{\alpha}}^{\lr{N}}}$, then
we must have that
\begin{equation}
  \label{eq:asymptotic-ionisation-overlap}
  \braket*
  {\vb{q}_{\alpha} \Phi_{n_{\alpha}}^{+}}
  {\Phi_{n}^{\lr{N}}}
  =
  \delta_{n_{\alpha}, n}
  \braket*
  {\vb{q}_{\alpha} \Phi_{n_{\alpha}}^{+}}
  {\Phi_{n}^{\lr{N}}}
  ,
\end{equation}
whence the ionisation amplitudes can be evaluated as
\begin{equation}
  \label{eq:scattering-amplitude-ionisation-evaluated}
  f_{n_{\alpha}, n_{i}}^{\lr{N}}\lr
  {
    \vb{k}_{\alpha}, \vb{q}_{\alpha}, \vb{k}_{i}
  }
  =
  \braket*
  {\vb{q}_{\alpha} \Phi_{n_{\alpha}}^{+}}
  {\Phi_{n_{\alpha}}^{\lr{N}}}
  \mel*
  {\vb{k}_{\alpha} \Phi_{n_{\alpha}}^{\lr{N}}}
  {\hat{T}}
  {\Phi_{n_{i}}^{\lr{N}} \vb{k}_{i}}
  ,
\end{equation}
at these $q_{\alpha}$ which satisfy
\eqref{eq:asymptotic-ionisation-energy-constraint}.

However, we note that a consequence of the assumed separability of the
asymptotic state in \eqref{eq:ccc-asymptotic-states} is that the
anti-symmetrisation of the asymptotic state is neglected.
Clearly this cannot be entirely neglected in the case of ionisation resulting in
two unbounded electron states, even if one is screened by the other.
Inclusion of the anti-symmetrisation of the ionised asymptotic state, with
respect to the two unbounded electron states, results in the transformation
\begin{equation}
  \label{eq:asymptotic-ionised-anti-symmetrised}
  \ket*{\Phi_{n_{\alpha}}^{+} \vb{q}_{\alpha} \vb{k}_{\alpha}}
  \mapsto
  \lrsq{1 - \hat{P}_{0, n_{\mathrm{e}}}}
  \ket*{\Phi_{n_{\alpha}}^{+} \vb{q}_{\alpha} \vb{k}_{\alpha}}
  =
  \ket*{\Phi_{n_{\alpha}}^{+} \vb{q}_{\alpha} \vb{k}_{\alpha}}
  -
  \exp{\imath \theta_{\alpha}}
  \ket*{\Phi_{n_{\alpha}}^{+} \vb{k}_{\alpha} \vb{q}_{\alpha}}
  ,
\end{equation}
where $\theta_{\alpha} \in \lrset{0, \pi}$ is the exchange phase, corresponding
to the exchange of the projectile and ionised electron states.
Whence, as described in \cite{PhysRevLett.78.4721, PhysRevLett.83.1570,
  PhysRevLett.89.273201}, we perform an ad-hoc anti-symmetrisation of the
ionisation amplitude to account for this, resulting in a corrected ionisation
amplitude $F_{n_{\alpha}, n_{i}}^{\lr{N}}$ of the form
\begin{equation}
  \label{eq:scattering-amplitude-ionisation-anti-symmetrised}
  F_{n_{\alpha}, n_{i}}^{\lr{N}}\lr
  {
    \vb{k}_{\alpha}, \vb{q}_{\alpha}, \vb{k}_{i}
  }
  =
  f_{n_{\alpha}, n_{i}}^{\lr{N}}\lr
  {
    \vb{k}_{\alpha}, \vb{q}_{\alpha}, \vb{k}_{i}
  }
  -
  \exp{- \imath \theta_{\alpha}}
  f_{n_{\alpha}, n_{i}}^{\lr{N}}\lr
  {
    \vb{q}_{\alpha}, \vb{k}_{\alpha}, \vb{k}_{i}
  }
  ,
\end{equation}
which satisfies
\begin{equation}
  \label{eq:scattering-amplitude-ionisation-anti-symmetrised-constraint}
  F_{n_{\alpha}, n_{i}}^{\lr{N}}\lr
  {
    \vb{k}_{\alpha}, \vb{q}_{\alpha}, \vb{k}_{i}
  }
  =
  -
  \exp{- \imath \theta_{\alpha}}
  F_{n_{\alpha}, n_{i}}^{\lr{N}}\lr
  {
    \vb{q}_{\alpha}, \vb{k}_{\alpha}, \vb{k}_{i}
  }
  .
\end{equation}
We note that in the CCC method, we refer to $f_{\alpha, i}^{\lr{N}}$ simply as
the ionisation amplitudes (or as the unsymmetrised ionisation amplitudes when
specificity is required), and we refer to $F_{\alpha, i}^{\lr{N}}$ as the
anti-symmetrised ionisation amplitudes.
We note that while the anti-symmetrised ionisation amplitudes are used for
comparison with experimental results, we make reference to the unsymmetrised
ionisation amplitudes in the discussion of ionisation in the CCC method.
Lastly, we note that we are constrained to evaluating these amplitudes only for
a countable number of outgoing projectile energies, bound by the constraint
defined in \eqref{eq:asymptotic-ionisation-energy-constraint}.
Evaluating the ionisation scattering amplitudes at any other energy requires
an interpolation between these energies.

\subsubsection{Cross-Sections}
\label{sec:th-ccc-cs}

We present expressions for the partial and total cross sections, in a manner
similar to \cite[10]{BRAY19951}.
In general terms, the partial cross sections are of the form
\begin{equation}
  \label{eq:partial-cross-sections}
  \sigma_{\alpha, \beta}
  =
  \sigma_{\alpha, \beta}\lr{\vb{k}_{\alpha}, \vb{k}_{\beta}}
  =
  \dfrac{k_{\alpha}}{k_{\beta}}
  \lrabs{f_{\alpha, \beta}}^{2}
  =
  \dfrac{k_{\alpha}}{k_{\beta}}
  \lrabs
  {
    \mel*
    {\vb{k}_{\alpha} \Phi_{\alpha}}
    {\hat{T}}
    {\Phi_{\beta} \vb{k}_{\beta}}
  }^{2}
  ,
\end{equation}
with the specific notation for elastic, discrete excitation, and ionisation
cross sections paralleling the notation used in
\eqref{eq:scattering-amplitude-discrete},
\eqref{eq:scattering-amplitude-elastic}, and
\eqref{eq:scattering-amplitude-ionisation} respectively.

The total cross section (TCS), for a given initial asymptotic state, is obtained
as a sum of all partial cross sections for which the outgoing asymptotic
projectile energy is positive,
\begin{equation}
  \label{eq:total-cross-section}
  \sigma_{\mathrm{T}; i}^{\lr{N}}
  =
  \sum_{f : k_{f} > 0}
  \sigma_{f, i}^{\lr{N}}
  ,
\end{equation}
while the total ionisation cross section (TICS), for a given initial asymptotic
state, is obtained as a sum of all partial cross sections for which the outgoing
asymptotic projectile and target energies are positive (and thus unbounded),
\begin{equation}
  \label{eq:total-ionisation-cross-section}
  \sigma_{\mathrm{I}; i}^{\lr{N}}
  =
  \sum_{\alpha : k_{\alpha} > 0, \epsilon_{\alpha}^{\lr{N}} > 0}
  \sigma_{\alpha, i}^{\lr{N}}
  .
\end{equation}
An ionisation cross section can also be constructed for a particular outgoing
asymptotic ionised target state by an appropriate
restriction of the sum in \eqref{eq:total-ionisation-cross-section},
\begin{equation}
  \label{eq:partial-ionisation-cross-section}
  \sigma_{\mathrm{I}; n_{f}, i}^{\lr{N}}
  =
  \sum_{\alpha :
    k_{\alpha} > 0
    ,
    \epsilon_{\alpha}^{\lr{N}} > 0
    ,
    n_{\alpha} = n_{f}
  }
  \sigma_{\alpha, i}^{\lr{N}}
  .
\end{equation}

We also consider the various differential cross sections in the context of
ionisation transitions, following in the form of \cite{PhysRevA.54.2991}.
Evaluating the partial cross sections, for an ionisation transition, yields the
triple-differential cross section (TDCS),
\begin{equation}
  \label{eq:triple-differential-cross-section}
  \dfrac
  {
    \dd{\sigma_{\alpha, i}^{\lr{N}}}
  }
  {
    \dd{\Omega_{k_{\alpha}}}
    \dd{\Omega_{q_{\alpha}}}
    \dd{e_{q_{\alpha}}}
  }
  \lr
  {
    \vb{k}_{\alpha}, \vb{q}_{\alpha}, \vb{k}_{i}
  }
  =
  \dfrac{k_{\alpha} q_{\alpha}}{k_{i}}
  \lrabs
  {
    F_{n_{\alpha}, n_{i}}^{\lr{N}}\lr
    {
      \vb{k}_{\alpha}, \vb{q}_{\alpha}, \vb{k}_{i}
    }
  }^{2}
  ,
\end{equation}
where $e_{q_{\alpha}} = \tfrac{1}{2} q_{\alpha}^{2} \in [0, E - \epsilon_{n_{\alpha}}^{+}]$ is the energy of the outgoing
projectile electron, and where $\Omega = \lr{\theta, \phi}$ refers to the
spherical coordinates of momentum-space.
Integrating the TDCS over the spherical coordinates of either the outgoing
asymptotic projectile electron, or the outgoing ionised target electron, yields
the double-differential cross section (DDCS).
Furthermore, integrating the DDCS over the spherical coordinates of the
remaining electron, whichever one that may be, yields the single-differential
cross section (SDCS), which is of the form
\begin{equation}
  \label{eq:single-differential-cross-section}
  \dv{\sigma_{\alpha, i}^{\lr{N}}}{e_{q_{\alpha}}}\lr{e_{q_{\alpha}}}
  =
  \dfrac{k_{\alpha} q_{\alpha}}{k_{i}}
  \int_{S^{2}}
  \dd{\Omega_{k_{\alpha}}}
  \int_{S^{2}}
  \dd{\Omega_{q_{\alpha}}}
  \lrabs
  {
    F_{n_{\alpha}, n_{i}}^{\lr{N}}\lr
    {
      \vb{k}_{\alpha}, \vb{q}_{\alpha}, \vb{k}_{i}
    }
  }^{2}
  ,
\end{equation}
where we recall that the energies of the incoming and outgoing projectile
states, as well as the ionised electron state, are constrained to be on-shell as
specified in \eqref{eq:asymptotic-ionised-energy}.
Integration of the SDCS over the projectile (or target) electron energy yields
the total ionisation cross section.

\subsection{Considerations for a Helium Target}
\label{sec:th-he-target}

We briefly remark on some of the specific considerations needed for a helium
target, in the CCC method.
We recall, from \autoref{eq:target-slater-determinant-basis}, that the target
psuedostates $\lrset{\ket*{\Phi_{n}^{\lr{N}}}}_{n = 1}^{N_{T}}$ are constructed
by expanding the target Hamiltonian $\hat{H}_{T}$ in a basis, formed from a
selection of Slater determinants, and diagonalising.
For the case of a helium target, these Slater determinants assume the form
\begin{equation}
  \label{eq:he-target-slater-determinant-basis}
  \lrset[\big]
  {
    \ket*{\chi_{\lrsq{a_{1}, a_{2}}}^{\lr{N}}}
    :
    a_{1}, a_{2}
    \in
    \lrset{1, \dotsc, 2N}
  }
  ,
\end{equation}
and we recall their anti-symmetric properties,
\begin{equation*}
  \ket*{\chi_{\lrsq{a_{2}, a_{1}}}^{\lr{N}}}
  =
  -
  \ket*{\chi_{\lrsq{a_{1}, a_{2}}}^{\lr{N}}}
  \qq{,}
  \ket*{\chi_{\lrsq{a_{1}, a_{1}}}^{\lr{N}}}
  =
  0
\end{equation*}
for $a_{1}, a_{2} \in \lrset{1, \dotsc, 2N}$.
The helium target pseudostates will be expressed in the form
\begin{equation}
  \label{eq:he-target-states-expansion}
  \ket*{\Phi_{n}^{\lr{N}}}
  =
  \sum_{a_{1} = 1}^{2N}
  \sum_{a_{2} > a_{1}}^{2N}
  D_{n}^{a_{1}, a_{2}}
  \ket*{\chi_{\lrsq{a_{1}, a_{2}}}^{\lr{N}}}
\end{equation}
for $n = 1, \dotsc, N_{T}$, where $D_{n}^{a_{1}, a_{2}}$ are the coefficients of
the expansion.
As mentioned earlier in \autoref{sec:th-ccc-target-diag}, we restrict our
selection of Slater determinants to those for which the first (core) orbital
ranges over a smaller set of spin orbitals,
$\lrset{\ket*{\chi_{i}^{\lr{N}}}}_{i = 1}^{2C}$ defined by the parameter
$C \leq N$, while placing no such restriction on the second (valence) orbital,
yielding target pseudostates of the form
\begin{equation}
  \label{eq:he-target-states-expansion-restricted}
  \ket*{\Phi_{n}^{\lr{C, N}}}
  =
  \sum_{a_{1} = 1}^{2C}
  \sum_{a_{2} > a_{1}}^{2N}
  D_{n}^{a_{1}, a_{2}}
  \ket*{\chi_{\lrsq{a_{1}, a_{2}}}^{\lr{N}}}
  .
\end{equation}
We also note that the target pseudostates, when expanded in the basis of Slater
determinants, are often dominated by a single term.
With this in mind, we say that the dominating Slater determinant is the major
configuration of a particular target pseudostate, and we define the major
configuration coefficient as the absolute value of the associated expansion
coefficient.

\clearpage

\section{Results}
\label{sec:re}

\subsection{Convergence Strategy}
\label{sec:re-conv}

\todo[inline]{
  Discuss which parameters are significant with the aim of attaining
  an accurate, convergent TICS.
}

With the aim of calculating convergent TICS for helium using the CCC method,
we first establish how convergence is to be achieved.
The parameters which most uniquely define our CCC calculations are:
\begin{itemize}
\item[$N$]
  the number of one-electron atomic orbitals used in the construction of the
  target pseudostates; equivalently, the number of Laguerre basis states,
\item[$C$]
  the number of core states; that is, the number of one-electron atomic orbitals
  that may be used for the core orbitals of the configurations used to
  construct the target pseudostates,
\item[$\lambda$]
  the exponential fall-off parameter of the Laguerre basis.
\end{itemize}
Hence, we denote a CCC calculation performed with a particular selection of
values for these parameters by $\mathrm{CCC}\lr{C, N, \lambda}$.

\todo[inline]{
  Discuss how convergence is attained in multi-parameter setting (increasing the
  number of core states for a fixed number of one-electron basis states).
}

We recall that the parameter $C \in \lrset{1, \dotsc, N}$ essentially acts to
restrict the number of configurations available for the representation of target
pseudostates as linear combinations of configurations.
We also recall that the parameter $C$ acts only on the core orbitals, while the
valence orbitals retain the use of all available one-electron atomic orbitals.
The projectile electron will interact with both the core and valence electrons
in the helium target during the scattering process, but we expect the valence
electron to play a more significant role in these interactions.
As a result, we expect that doubly-excited configurations with higher
excitations of the core orbital will have diminishing influence in the
scattering processes.
Furthermore, we expect the accuracy of the calculations to increase
diminishingly as the parameter $C$ increases.
Hence, for fixed values of the parameters $N$ and $\lambda$, we expect that the
calculations will converge with regard to parameter $C$, in the sense of
\begin{equation*}
  \mathrm{CCC}\lr{N, \lambda}
  =
  \lim_{C \to N}
  \mathrm{CCC}\lr{C, N, \lambda}
  .
\end{equation*}
We then sought to obtain convergence with regard to parameter $N$, with the role
of the parameter $\lambda$ initially assumed to be supplemental.
However, as discussed further in \autoref{sec:re-mixed}, the effect of these
parameters on the TICS calculations proved to be more nuanced.

\todo[inline]{
  Discuss increase in computational cost with increasing number of core states.
}

We also recall now that the number of target pseudostates used in the
calculation $\mathrm{CCC}\lr{C, N, \lambda}$ scales as
$\mathcal{O}\lr{N_{T}} = C N$, with the parameter $C \in \lrset{1, \dotsc, N}$.
As a consequence, it is often computationally impractical to perform more than a
few calculations of the form
$\lrset{\mathrm{CCC}\lr{C, N, \lambda} : C = 1, 2, \dotsc}$.
This is balanced by the fact that the TICS for these calculations tended to
converge sufficiently by $C \approx 6$.

\clearpage

\subsection{TICS-without-Excitation}
\label{sec:re-tics-iw}

\todo[inline]{
  Establish agreement between CCC and PECS calculations for
  TICS-without-excitation.
}

\todo[inline]{
  Figure of TICS-without-excitation, demonstrating agreement with PECS data.
}

% \begin{figure}[h]
%   \begin{center}
%     \input{figures/cs_tics_iw/20/figure.tex}
%   \end{center}
%   \caption[TICS-without-excitation: $\mathrm{CCC}\lr{C, 20, 0.50}$]{
%     Total cross sections for electron-impact ionisation-without-excitation of
%     helium, leaving the helium ion in a 1s state.
%     $\mathrm{CCC}\lr{C, N, \lambda}$ calculations, with $N = 20$ and
%     $\lambda = 0.50$, are presented for $C = 2, \dotsc, 6$, and are compared
%     with PECS \cite{PhysRevA.81.022716} calculations.
%   }
%   \label{fig:cs_tics_iw_20}
% \end{figure}

% \begin{figure}[h]
%   \begin{center}
%     \input{figures/cs_tics_iw/25/figure.tex}
%   \end{center}
%   \caption[TICS-without-excitation: $\mathrm{CCC}\lr{C, 25, 0.50}$]{
%     Total cross sections for electron-impact ionisation-without-excitation of
%     helium, leaving the helium ion in a 1s state.
%     $\mathrm{CCC}\lr{C, N, \lambda}$ calculations, with $N = 25$ and
%     $\lambda = 0.50$, are presented for $C = 2, \dotsc, 6$, and are compared
%     with PECS \cite{PhysRevA.81.022716} calculations.
%   }
%   \label{fig:cs_tics_iw_25}
% \end{figure}

% \begin{figure}[h]
%   \begin{center}
%     \input{figures/cs_tics_iw/30/figure.tex}
%   \end{center}
%   \caption[TICS-without-excitation: $\mathrm{CCC}\lr{C, 30, 0.50}$]{
%     Total cross sections for electron-impact ionisation-without-excitation of
%     helium, leaving the helium ion in a 1s state.
%     $\mathrm{CCC}\lr{C, N, \lambda}$ calculations, with $N = 30$ and
%     $\lambda = 0.50$, are presented for $C = 2, \dotsc, 6$, and are compared
%     with PECS \cite{PhysRevA.81.022716} calculations.
%   }
%   \label{fig:cs_tics_iw_30}
% \end{figure}

% \begin{figure}[h]
%   \begin{center}
%     \input{figures/cs_tics_iw/35/figure.tex}
%   \end{center}
%   \caption[TICS-without-excitation: $\mathrm{CCC}\lr{C, 35, 0.50}$]{
%     Total cross sections for electron-impact ionisation-without-excitation of
%     helium, leaving the helium ion in a 1s state.
%     $\mathrm{CCC}\lr{C, N, \lambda}$ calculations, with $N = 35$ and
%     $\lambda = 0.50$, are presented for $C = 2, \dotsc, 5$, and are compared
%     with PECS \cite{PhysRevA.81.022716} calculations.
%   }
%   \label{fig:cs_tics_iw_35}
% \end{figure}

\begin{figure}[h]
  \begin{center}
    \input{figures/cs_tics_iw/40/figure.tex}
  \end{center}
  \caption[TICS-without-excitation: $\mathrm{CCC}\lr{C, 40, 0.50}$]{
    Total cross sections for electron-impact ionisation-without-excitation of
    helium, leaving the helium ion in a 1s state.
    $\mathrm{CCC}\lr{C, N, \lambda}$ calculations, with $N = 40$ and
    $\lambda = 0.50$, are presented for $C = 2, \dotsc, 5$, and are compared
    with PECS \cite{PhysRevA.81.022716} calculations.
  }
  \label{fig:cs_tics_iw_40}
\end{figure}

% \begin{figure}[h]
%   \begin{center}
%     \input{figures/cs_tics_iw/50/figure.tex}
%   \end{center}
%   \caption[TICS-without-excitation: $\mathrm{CCC}\lr{C, 50, 0.50}$]{
%     Total cross sections for electron-impact ionisation-without-excitation of
%     helium, leaving the helium ion in a 1s state.
%     $\mathrm{CCC}\lr{C, N, \lambda}$ calculations, with $N = 50$ and
%     $\lambda = 0.50$, are presented for $C = 2, \dotsc, 4$, and are compared
%     with PECS \cite{PhysRevA.81.022716} calculations.
%   }
%   \label{fig:cs_tics_iw_50}
% \end{figure}

\clearpage

\subsection{TICS-with-Excitation}
\label{sec:re-tics-ie}

\todo[inline]{
  Discuss how convergence was demonstrated with an increasing number of core
  states, for a fixed number of one-electron basis states.
}

\todo[inline]{
  Figure(s) of TICS-with-excitation for increasing number of core states,
  demonstrating convergence.
}

\begin{figure}[h]
  \begin{center}
    \input{figures/cs_tics_ie_n/20/figure.tex}
  \end{center}
  \caption[TICS-with-excitation: $\mathrm{CCC}\lr{C, 20, 0.50}$]{
    Total cross sections for electron-impact ionisation-with-excitation of
    helium, leaving the helium ion in a 2s state.
    $\mathrm{CCC}\lr{C, N, \lambda}$ calculations, with $N = 20$ and
    $\lambda = 0.50$, are presented for $C = 2, \dotsc, 6$, and are compared
    with PECS \cite{PhysRevA.81.022716} calculations.
  }
  \label{fig:cs_tics_ie_n_20}
\end{figure}

\begin{figure}[h]
  \begin{center}
    \input{figures/cs_tics_ie_n/25/figure.tex}
  \end{center}
  \caption[TICS-with-excitation: $\mathrm{CCC}\lr{C, 25, 0.50}$]{
    Total cross sections for electron-impact ionisation-with-excitation of
    helium, leaving the helium ion in a 2s state.
    $\mathrm{CCC}\lr{C, N, \lambda}$ calculations, with $N = 25$ and
    $\lambda = 0.50$, are presented for $C = 2, \dotsc, 6$, and are compared
    with PECS \cite{PhysRevA.81.022716} calculations.
  }
  \label{fig:cs_tics_ie_n_25}
\end{figure}

\begin{figure}[h]
  \begin{center}
    \input{figures/cs_tics_ie_n/30/figure.tex}
  \end{center}
  \caption[TICS-with-excitation: $\mathrm{CCC}\lr{C, 30, 0.50}$]{
    Total cross sections for electron-impact ionisation-with-excitation of
    helium, leaving the helium ion in a 2s state.
    $\mathrm{CCC}\lr{C, N, \lambda}$ calculations, with $N = 30$ and
    $\lambda = 0.50$, are presented for $C = 2, \dotsc, 6$, and are compared
    with PECS \cite{PhysRevA.81.022716} calculations.
  }
  \label{fig:cs_tics_ie_n_30}
\end{figure}

\begin{figure}[h]
  \begin{center}
    \input{figures/cs_tics_ie_n/35/figure.tex}
  \end{center}
  \caption[TICS-with-excitation: $\mathrm{CCC}\lr{C, 35, 0.50}$]{
    Total cross sections for electron-impact ionisation-with-excitation of
    helium, leaving the helium ion in a 2s state.
    $\mathrm{CCC}\lr{C, N, \lambda}$ calculations, with $N = 35$ and
    $\lambda = 0.50$, are presented for $C = 2, \dotsc, 5$, and are compared
    with PECS \cite{PhysRevA.81.022716} calculations.
  }
  \label{fig:cs_tics_ie_n_35}
\end{figure}

\begin{figure}[h]
  \begin{center}
    \input{figures/cs_tics_ie_n/40/figure.tex}
  \end{center}
  \caption[TICS-with-excitation: $\mathrm{CCC}\lr{C, 40, 0.50}$]{
    Total cross sections for electron-impact ionisation-with-excitation of
    helium, leaving the helium ion in a 2s state.
    $\mathrm{CCC}\lr{C, N, \lambda}$ calculations, with $N = 40$ and
    $\lambda = 0.50$, are presented for $C = 2, \dotsc, 5$, and are compared
    with PECS \cite{PhysRevA.81.022716} calculations.
  }
  \label{fig:cs_tics_ie_n_40}
\end{figure}

\begin{figure}[h]
  \begin{center}
    \input{figures/cs_tics_ie_n/50/figure.tex}
  \end{center}
  \caption[TICS-with-excitation: $\mathrm{CCC}\lr{C, 50, 0.50}$]{
    Total cross sections for electron-impact ionisation-with-excitation of
    helium, leaving the helium ion in a 2s state.
    $\mathrm{CCC}\lr{C, N, \lambda}$ calculations, with $N = 50$ and
    $\lambda = 0.50$, are presented for $C = 2, \dotsc, 4$, and are compared
    with PECS \cite{PhysRevA.81.022716} calculations.
  }
  \label{fig:cs_tics_ie_n_50}
\end{figure}

\clearpage

\todo[inline]{
  Discuss how this series of convergent calculations failed to demonstrate
  convergence with regard to the increasing number of one-electron basis states.
}

Hence, the most natural way of obtaining convergence overall is to first obtain
convergence in $C$ for fixed values of $N$ and $\lambda$.
However, the suggestion that overall convergence might then be obtained by
simply increasing $N$ seems natural, but proved to be ineffectual.
While a larger basis of one-electron atomic orbitals might be expected to
facilitate increasingly accurate target pseudostates, and indeed it might, it
also provides more opportunities for the complicated auto-ionising states of
helium to interfere.

\todo[inline]{
  Discuss the susceptibility of TICS-with-excitation to small variations in the
  exponential fall-off parameter used when constructing the one-electron basis
  states.
}

\todo[inline]{
  Figure(s) of TICS-with-excitation for varying exponential fall-off parameter
  values demonstrating significant variation.
}

\begin{figure}[h]
  \begin{center}
    \input{figures/cs_tics_ie_alpha/2_35/figure.tex}
  \end{center}
  \caption[TICS-with-excitation: $\mathrm{CCC}\lr{2, 35, \lambda}$]{
    Total cross sections for electron-impact ionisation-with-excitation of
    helium, leaving the helium ion in a 2s state.
    $\mathrm{CCC}\lr{C, N, \lambda}$ calculations, with $C = 2$ and $N = 35$,
    are presented for $\lambda = 0.40, 0.41, \dotsc, 0.65$, and are compared
    with PECS \cite{PhysRevA.81.022716} calculations.
  }
  \label{fig:cs_tics_ie_alpha}
\end{figure}

\clearpage

\subsection{Mixed Target States}
\label{sec:re-mixed}

\todo[inline]{
  Raise the possibility of strongly-mixed target states causing inaccurate
  TICS-with-excitation calculations.
}

\todo[inline]{
  Discuss the difference in magnitude between elastic, TICS-without-excitation,
  and TICS-with-excitation calculations.
}

\todo[inline]{
  Figure of elastic, TICS-without-excitation, and TICS-with-excitation
  calculations, demonstrating differences in magnitude.
}

\begin{figure}[h]
  \begin{center}
    \input{figures/cs/35/figure.tex}
  \end{center}
  \caption[TICS-with-excitation: $\mathrm{CCC}\lr{C, 35, 0.50}$]{
    Total cross sections for electron-impact ionisation-without-excitation
    (1sks) and ionisation-with-excitation (2sks) of helium.
    $\mathrm{CCC}\lr{C, N, \lambda}$ calculations, with $C = 5$, $N = 35$ and
    $\lambda = 0.50$, are presented and compared with PECS
    \cite{PhysRevA.81.022716} calculations.
    Note that for clarity, the ionisation-with-excitation (2sks) cross sections
    are presented scaled up by a factor of 10.
  }
  \label{fig:cs_35}
\end{figure}

\clearpage

\todo[inline]{
  Discuss the overlapping nature of the continuous and discrete parts of
  the Helium's energy spectrum, and the implications this has with regard to
  the mixing of target states.
}

Where the pseudoenergies tend to a constant value over a range of $\lambda$
values, they are approximating the energy levels of discrete helium states.
On the other hand, where the pseudoenergies tend to vary over a range of
$\lambda$ values, they correspond to pseudostates which are discretisations of
the continuum helium states.

\todo[inline]{
  Figure(s) of Helium energy spectrum for varying exponential fall-off parameter
  values, examining the separation of target state energies.
}

\begin{figure}[h]
  \begin{center}
    \input{figures/he_state/2_35/singlet/figure.tex}
  \end{center}
  \caption[Singlet Helium Pseudoenergies]{
    The spectrums of pseudoenergies corresponding to singlet helium
    pseudostates, calculated with $C = 2$ and $N = 35$, are presented for
    $\lambda = 0.40, 0.41, \dotsc, 0.65$.
  }
  \label{fig:he_state_singlet}
\end{figure}

\begin{figure}[h]
  \begin{center}
    \input{figures/he_state/2_35/triplet/figure.tex}
  \end{center}
  \caption[Triplet Helium Pseudoenergies]{
    The spectrums of pseudoenergies corresponding to triplet helium
    pseudostates, calculated with $C = 2$ and $N = 35$, are presented for
    $\lambda = 0.40, 0.41, \dotsc, 0.65$.
  }
  \label{fig:he_state_triplet}
\end{figure}

\begin{figure}[h]
  \begin{center}
    \input{figures/he_state/2_35/singlet_mixing/figure.tex}
  \end{center}
  \caption[Singlet Helium Pseudoenergies - Auto-Ionising Region]{
    The same data is shown as in \autoref{fig:he_state_singlet}, with the y-axis
    restricted to focus on the auto-ionising region, wherein discrete
    doubly-excited states overlap with continuum states.
    For clarity, a number of pseudoenergies are annotated by their corresponding
    major configuration across ranges of $\lambda$ values, including discrete
    doubly-excited states (2s2s, 2s3s) and continuum states (1s29s, 1s30s,
    1s31s).
  }
  \label{fig:he_state_singlet_auto_ionising}
\end{figure}

\clearpage

\todo[inline]{
  Discuss the correspondence between the purity of a target state and how
  well-separated it's energy is within the spectrum.
}

\todo[inline]{
  Figure(s) of major-configuration coefficient of various target states, for
  varying exponetial fall-off parameter values, demonstrating the correspondence
  between energy-separability and mixing.
}

\begin{figure}[h]
  \begin{center}
    \input{figures/mcc/2_35/singlet/3/figure.tex}
  \end{center}
  \caption[Major Configuration Coefficients: Singly-Excited]{
    The major configuration coefficients of the singlet helium pseudostates,
    $\ket*{\Phi_{n}^{\lr{C, N}}}$ calculated with $C = 2$ and $N = 35$, are
    presented across a range of $\lambda = 0.40, 0.41, \dotsc, 0.65$, for the
    states $n = 2, 3, 4$.
  }
  \label{fig:mcc_3}
\end{figure}

\begin{figure}[h]
  \begin{center}
    \input{figures/mcc/2_35/singlet/30/figure.tex}
  \end{center}
  \caption[Major Configuration Coefficients: Auto-Ionising I]{
    The major configuration coefficients of the singlet helium pseudostates,
    $\ket*{\Phi_{n}^{\lr{C, N}}}$ calculated with $C = 2$ and $N = 35$, are
    presented across a range of $\lambda = 0.40, 0.41, \dotsc, 0.65$, for the
    states $n = 29, 30, 31$.
    Note that we present data points as being connected when they are associated
    with the same major configuration.
    Note also that blue lines indicate that the major configuration has an
    unexcited core state (1s), while red lines indicate an excited core state
    (2s).
  }
  \label{fig:mcc_30}
\end{figure}

\begin{figure}[h]
  \begin{center}
    \input{figures/mcc/2_35/singlet/34/figure.tex}
  \end{center}
  \caption[Major Configuration Coefficients: Auto-Ionising II]{
    The major configuration coefficients of the singlet helium pseudostates,
    $\ket*{\Phi_{n}^{\lr{C, N}}}$ calculated with $C = 2$ and $N = 35$, are
    presented across a range of $\lambda = 0.40, 0.41, \dotsc, 0.65$, for the
    states $n = 33, 34, 35$.
    Note that we present data points as being connected when they are associated
    with the same major configuration.
    Note also that blue lines indicate that the major configuration has an
    unexcited core state (1s), while red lines indicate an excited core state
    (2s).
  }
  \label{fig:mcc_34}
\end{figure}

\begin{figure}[h]
  \begin{center}
    \input{figures/mcc/2_35/singlet/53/figure.tex}
  \end{center}
  \caption[Major Configuration Coefficients: Excited-plus-Ionised]{
    The major configuration coefficients of the singlet helium pseudostates,
    $\ket*{\Phi_{n}^{\lr{C, N}}}$ calculated with $C = 2$ and $N = 35$, are
    presented across a range of $\lambda = 0.40, 0.41, \dotsc, 0.65$, for the
    states $n = 52, 53, 54$.
    Note that we present data points as being connected when they are associated
    with the same major configuration.
    Note also that blue lines indicate that the major configuration has an
    unexcited core state (1s), while red lines indicate an excited core state
    (2s).
  }
  \label{fig:mcc_53}
\end{figure}

\clearpage

\todo[inline]{
  Discuss how the mixing of target states permits the TICS-with-excitation to
  contain contributions from target states which have significant
  ionised-without-excitation components, and that these contributions are of a
  generally larger magnitude.
}

\todo[inline]{
  Figure(s) of partial cross sections of various target states, for varying
  exponential fall-off parameter values, demonstrating how the level of mixing
  influences the magnitude of the cross sections.
}

\begin{figure}[h]
  \begin{center}
    \input{figures/pcs/2_35/singlet/3/figure.tex}
  \end{center}
  \caption[Partial Cross Sections: Singly-Excited]{
    The partial cross sections of ground state singlet helium to the singlet
    pseudostates, $\ket*{\Phi_{n}^{\lr{C, N}}}$ calculated with $C = 2$ and
    $N = 35$, are presented across a range of
    $\lambda = 0.40, 0.41, \dotsc, 0.65$, for the states $n = 2, 3, 4$.
    Note that these cross sections have been calculated over a range of
    projectile electron energies, \SIrange{0}{500}{\eV}, but are presented
    orthographically, obscuring this axis; effectively we present the maximum of
    each cross section across this range of projectile electron energies.
  }
  \label{fig:pcs_3}
\end{figure}

\begin{figure}[h]
  \begin{center}
    \input{figures/pcs/2_35/singlet/30/figure.tex}
  \end{center}
  \caption[Partial Cross Sections: Auto-Ionising I]{
    The partial cross sections of ground state singlet helium to the singlet
    pseudostates, $\ket*{\Phi_{n}^{\lr{C, N}}}$ calculated with $C = 2$ and
    $N = 35$, are presented across a range of
    $\lambda = 0.40, 0.41, \dotsc, 0.65$, for the states $n = 29, 30, 31$.
    Note that these cross sections have been calculated over a range of
    projectile electron energies, \SIrange{0}{500}{\eV}, but are presented
    orthographically, obscuring this axis; effectively we present the maximum of
    each cross section across this range of projectile electron energies.
  }
  \label{fig:pcs_30}
\end{figure}

\begin{figure}[h]
  \begin{center}
    \input{figures/pcs/2_35/singlet/34/figure.tex}
  \end{center}
  \caption[Partial Cross Sections: Auto-Ionising II]{
    The partial cross sections of ground state singlet helium to the singlet
    pseudostates, $\ket*{\Phi_{n}^{\lr{C, N}}}$ calculated with $C = 2$ and
    $N = 35$, are presented across a range of
    $\lambda = 0.40, 0.41, \dotsc, 0.65$, for the states $n = 33, 34, 35$.
    Note that these cross sections have been calculated over a range of
    projectile electron energies, \SIrange{0}{500}{\eV}, but are presented
    orthographically, obscuring this axis; effectively we present the maximum of
    each cross section across this range of projectile electron energies.
  }
  \label{fig:pcs_34}
\end{figure}

\begin{figure}[h]
  \begin{center}
    \input{figures/pcs/2_35/singlet/53/figure.tex}
  \end{center}
  \caption[Partial Cross Sections: Excited-plus-Ionised]{
    The partial cross sections of ground state singlet helium to the singlet
    pseudostates, $\ket*{\Phi_{n}^{\lr{C, N}}}$ calculated with $C = 2$ and
    $N = 35$, are presented across a range of
    $\lambda = 0.40, 0.41, \dotsc, 0.65$, for the states $n = 52, 53, 54$.
    Note that these cross sections have been calculated over a range of
    projectile electron energies, \SIrange{0}{500}{\eV}, but are presented
    orthographically, obscuring this axis; effectively we present the maximum of
    each cross section across this range of projectile electron energies.
  }
  \label{fig:pcs_53}
\end{figure}

\clearpage

\section{Discussion}
\label{sec:dis}

\clearpage

\section{Conclusions}
\label{sec:co}

\clearpage

\bibliography{references}

\clearpage

\listoftodos

\end{document}
